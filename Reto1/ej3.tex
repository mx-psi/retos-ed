\begin{ejercicio}{
3. Ordenar la siguiente lista de acuerdo a la notación $O$}

Tomamos como base
$1 \leq \log n \leq \sqrt{n} \leq n \leq n \log n \leq n^2 \leq n^3 \leq 2^n$

\vspace*{10mm}

\begin{description}[
  font=$\bullet$ \scshape\normalfont\mathversion{bold}]

\item[$2^{n} \leq 2^{2^n}$:]
$n \leq 2^n \implies  2^n \leq 2^{2^n}$, con $c = 2^c$.

\item[$n^3 \leq 2^{n}$]

\item[$4n^{\frac{3}{2}} \leq n^3$:]
$n^{\frac{3}{2}} = n\sqrt{n}$, por lo que:
$\sqrt{n} \leq n \implies
n^2 \cdot \sqrt{n} \leq n\cdot n^2$  con el mismo $c$

\item[$2 n \log^2 n \leq 4n^{\frac{3}{2}}$:]

\item[$6n \log_2 n \leq 2 n \log^2 n$:]

\item[$6n \log_2 n \equiv n \log_4 n$:]
Como $\log_4 n = \frac{\log_2 n}{\log_2 4}$ sólo difieren de una constante.

\item[$5n \leq 6n \log_2 n$:]
Se sigue de $n \leq n \log n$ ya que sólo difieren en constantes.

\item[$2^{\log n} \equiv 5n$:]
$2^{\log n} = n$, por lo que los términos sólo difieren en una constante.

\item[$\sqrt{n} \leq 5n $:]
Se sigue de $\sqrt{n} \leq n$, ya que sólo difieren en constantes

\item[$3n^{0,5} \equiv \sqrt{n}$:]
$3n^{0.5} = 3\sqrt{n}$ que sólo difiere en la constante.

\item[$n^{0,01} \leq \sqrt{n} $:]


\item[$\log^2 n \leq n^{0,01}$:]

\item[$\log \log n \leq \log^2 n $:]

\item[$\frac{1}{n} \leq \log \log n $:]

\end{description}
\end{ejercicio}
