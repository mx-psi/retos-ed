\begin{ejercicio}
{2. Determinar la eficiencia de la siguiente función:}
\begin{lstlisting}[
language=C++, directivestyle={\color{black}},
emph={int,char,double,float,unsigned}]
int cualeslaeficiencia(bool existe)
{
  int sum2=0; int k,j,n;
  if(existe)
    for(k=1; k<=n; k*=2)
      for(j=1; j<=k; j++)
        sum2++;
  else
    for(k=1; k<=n; k*=2)
      for(j=1; j<=n; j++)
        sum2++;
  return sum2;
}
\end{lstlisting}
\vspace*{5mm}

Primero analizamos los dos bucles del condicional:

\begin{itemize}

\item El \textbf{primer bucle} (\n{5-7}) se repite $\log_2(n)$ veces,
ya que $k$ se  multiplica por 2 cada vez.
El bucle interior (\n{6-7}) repite una sentencia $O(1)$ $k$ veces.
Cada $k$ puede escribirse de la forma $2^i$ para $i \in \mathbb{N}$.
Es decir, el número de pasos del bucle será:

\[\sum_{i=0}^{\log_2(n)} 2^i = 2^{\log_2{n}} -1 = n -1 \in O(n)\]

\item El \textbf{segundo bucle} (\n{9-11}) se repite $\log_2(n)$ veces.
El bucle interior (\n{10-11}) repite una sentencia $O(1)$ $n$ veces.
Es decir, el número de pasos del bucle será, por la regla del producto:
$O(n\log(n))$
\end{itemize}

Como estamos ante un condicional tomamos el máximo de ambos,
obteniendo que la eficiencia es:

\[O(1) + O(\max\{n,n\log n\}) + O(1)  = \resultado{O(n\log n)}\]

\end{ejercicio}
