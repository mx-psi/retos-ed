\documentclass{article}
\usepackage{multirow}
\usepackage[spanish]{babel}
\usepackage{lmodern}
\usepackage{amssymb,amsmath,amsthm}
\usepackage[spanish,onelanguage]{algorithm2e}
\usepackage[T1]{fontenc}
\usepackage[utf8]{inputenc}
\usepackage{amsfonts,dsfont}
\usepackage{enumitem}
\usepackage[margin=1in]{geometry}

\title{Reto 4}
\date{Estructura de Datos}
\author{Pablo Baeyens Fernández\\José Manuel Muñoz Fuentes\\Darío Sierra Martínez}

\begin{document}
\maketitle

\section{Introducción}

% Diría que falta más texto, por lo menos, antes de lo siguiente.

Propondremos una estructura de fichero que tendrá el menor tamaño en disco
posible cumpliendo las siguientes premisas:
\begin{itemize}
	\item \textbf{La reconstrucción del árbol es correcta y unívoca.}
  El procedimiento no debe guardar dos archivos distintos al aplicarlo a un mismo
  árbol, y dos archivos iguales deben generar al mismo árbol.
	\item \textbf{Los datos se guardan en binario, y no como texto.} Las etiquetas
  del árbol se almacenarán en el archivo de forma similar a como lo hacen en
  memoria. Si son un tipo de dato simple, se guardarán como tal. Si son un puntero
  a un vector de datos que terminan en un elemento terminador, se guardarán
  estos elementos (incluyendo el terminador). Si son una estructura o un vector
  de datos de tamaño fijo, se guardarán como la yuxtaposición de los elementos
  de esa estructura.
	\item \textbf{Las etiquetas no se modifican.} No se aprovechará el contenido
  de las etiquetas para, modificando bits que no sean relevantes, introducir
  información en ellas. Por ejemplo, si el tipo de dato almacenado son enteros
  sin signo de $32$ bits y se sabe que ningún elemento es mayor que $2^{30}-1$,
  no se aprovecharán los dos primeros bits para incluir información en ellos.
	\item \textbf{No se aplica ningún algoritmo de compresión a los datos.}
  Existen algoritmos de compresión de datos que podrían ayudar a reducir el
  tamaño de los archivos, tanto en las etiquetas como en los datos adicionales
  que se incluyan en el fichero. Supondremos que su uso queda fuera del interé
  s de este reto.
	\item \textbf{No se incluyen bits que identifiquen el tipo de archivo
  o garanticen su integridad.} No habrá cabecera en los archivos, ni se usará un
  número mágico, ni se añadirán bytes para forzar un tipo de suma de verificación
  con un valor concreto. Esto también queda fuera de las intenciones del reto.
  Por ello, será responsabilidad del operario saber si el archivo que está
  intentando abrir como árbol es realmente un árbol guardado con este formato.
\end{itemize}

\section{Estructura de fichero}

El archivo se compondrá de la yuxtaposición de una \textbf{clave} y un
bloque de \textbf{datos}.

\begin{itemize}
	\item La \textbf{clave} será una lista de bits de longitud indefinida.
  Para cada nodo del árbol se añadirá un 1 a la lista si el nodo tiene un hijo a
  la izquierda o un 0 en caso contrario, y a continuación se añadirá un 1 si el
  nodo tiene un hijo a la derecha o un 0 en caso contrario. Este procedimiento
  se repite con los nodos hijos (primero el de la izquierda y después el de la
  derecha), de forma que se almacena el par de bits de cada nodo en preorden.
  Para que la siguiente sección no comience en una posición intermedia de un byte,
  si al escribir la lista de bits en disco el número de elementos no es un múltiplo
  de 8, se añadirán ceros al final hasta que lo sea. (Esto en ningún caso
  incrementará el tamaño del archivo, puesto que, como el próximo bloque siempre
  tiene por tamaño un múltiplo de un byte, el sistema operativo añadiría bits al
  final del archivo hasta que el tamaño fuese un múltiplo de un byte)
	\item El bloque de \textbf{datos} lo formará el grupo de datos, almacenados
  en preorden sin espacios entre ellos. La posición donde empieza un nuevo dato
  se determinará a partir del tamaño de los tipos de datos (si son de tamaño fijo)
  o de la posición de un elemento terminador (si son de tamaño variable).
\end{itemize}

\subsection{Lectura}

\subsection{Escritura}

\end{document}
