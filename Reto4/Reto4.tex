\documentclass{article}
\usepackage{multirow}
\usepackage[spanish]{babel}
\usepackage{lmodern}
\usepackage{amssymb,amsmath,amsthm}
\usepackage[spanish,onelanguage]{algorithm2e}
\usepackage[T1]{fontenc}
\usepackage[utf8]{inputenc}
\usepackage{amsfonts,dsfont}
\usepackage{enumitem}
\usepackage[margin=1in]{geometry}

\newtheorem{prop}{Proposición}

% Entorno para controlar el espacio antes y después de un algoritmo
\newenvironment{algo}{
\vspace*{0.5cm}
\begin{algorithm}[H]}{
\end{algorithm}
\vspace*{0.5cm}
}

% Para los algoritmos
\SetKwFunction{GuardaBits}{GuardaBits}
\SetKwFunction{AvanzaArmazon}{AvanzaArmazon}

\title{Reto 4}
\date{Estructura de Datos}
\author{Pablo Baeyens Fernández\\José Manuel Muñoz Fuentes\\Darío Sierra Martínez}

\begin{document}
\maketitle

\section{Introducción}

En este reto se propone la obtención de un procedimiento para guardar un árbol
binario de estructura y tipo de dato arbitrarios en disco y otro procedimiento
para reconstruir correctamente ese árbol a partir de sus datos en disco, de forma
que el tamaño en disco del árbol sea el mínimo posible. \\

\subsection{Problema}

Propondremos una estructura de fichero que tendrá el menor tamaño en disco
posible cumpliendo las siguientes premisas:
\begin{itemize}
	\item \textbf{La reconstrucción del árbol es correcta y unívoca.}
  El procedimiento no debe guardar dos archivos distintos al aplicarlo a un mismo
  árbol, y dos archivos iguales deben generar al mismo árbol.
	\item \textbf{Los datos se guardan en binario, y no como texto.} Las etiquetas
  del árbol se almacenarán en el archivo de forma similar a como lo hacen en
  memoria. Si son un tipo de dato simple, se guardarán como tal. Si son un puntero
  a un vector de datos que concluye en un elemento terminador, se guardarán estos
  elementos (incluyendo el terminador). Si son una estructura o un vector de datos
  de tamaño fijo, se guardarán como la yuxtaposición de los elementos de esa
  estructura.
	\item \textbf{Las etiquetas no se modifican.} No se aprovechará el contenido
  de las etiquetas para, modificando bits que no sean relevantes, introducir
  información en ellas. Por ejemplo, si el tipo de dato almacenado son enteros
  sin signo de $32$ bits y se sabe que ningún elemento es mayor que $2^{30}-1$,
  no se aprovecharán los dos primeros bits para incluir información en ellos.
	\item \textbf{No se aplica ningún algoritmo de compresión a los datos.}
  Existen algoritmos de compresión de datos que podrían ayudar a reducir el
  tamaño de los archivos, tanto en las etiquetas como en los datos adicionales
  que se incluyan en el fichero. Supondremos que su uso queda fuera del interés
  de este reto.
	\item \textbf{No se incluyen bits que identifiquen el tipo de archivo
  o garanticen su integridad.} No habrá cabecera en los archivos, ni se usará un
  número mágico, ni se añadirán bytes para forzar un tipo de suma de verificación
  con un valor concreto. Esto también queda fuera de las intenciones del reto.
  Por ello, será responsabilidad del operario saber si el archivo que está
  intentando abrir como árbol es realmente un árbol guardado con este formato.
\end{itemize}

Para ello daremos una estructura que prescindirá de centinelas y en su lugar
almacenará al principio del archivo una \textbf{clave} que identificará
unívocamente la estructura del árbol. Así, el reto se reduce a describir dos
algoritmos que computen las aplicaciones $c: A \to B$, $d: c(A) \to A$, donde:

\begin{itemize}
	\item $A$ es el conjunto de los árboles binarios de un número finito de nodos.
	\item $B$ es el conjunto de las cadenas de longitud finita formadas por $0$ y
	$1$.
	\item $c$ y $d$ son aplicaciones inyectivas.
	\item $d \circ c = 1_A$, es decir, $d$ es una inversa por la izquierda de $c$.
	\item Para todo árbol $a \in A$, $|c(a)|$ (la longitud de la cadena) es mínima.
\end{itemize}

\subsection{Cotas}

En esta sección ofrecemos una cota inferior al problema. Sea $m \in \mathbb{N}$.
Mediante un argumento combinatorio es fácil mostrar que:
\[|\{b \in B\; : \; |b| = m\}| = 2^m\]

ya que son las combinaciones de 0 y 1 con $m$ elementos y con repetición. De esta
forma, podemos ver que si tenemos una familia de $n$ objetos distintos, podremos
etiquetarlos con etiquetas binarias con una longitud mínima de
$\lceil \log_2(n) \rceil$. \\

El número de árboles binarios de $n$ nodos viene dado por la sucesión de los números
de Catalan:

\[C_n = \frac{1}{n+1} {2n\choose n} = \frac{(2n)!}{n!(n+1)!} \]

Puede probarse que $C_n \le 4^n \ \ \forall n \in \mathbb N$. Para $C_0$ es
cierto: $C_0 = 1 = 4^0$. Sea cierto que $C_n \le 4^n$ para un $n \in \mathbb N$.
Se observa que $C_{n+1} < 4C_n \ \ \forall n \in \mathbb N$ puesto que:
\[\frac{C_{n+1}}{C_n} = \frac{(2n+2)!}{(n+1)!(n+2)!} \frac{n!(n+1)!}{(2n)!}
= \frac{(2n+2)(2n+1)}{(n+2)(n+1)} = \frac{2(n+1)(2n+1)}{(n+2)(n+1)} =
\frac{2(2n+1)}{(n+2)} < \frac{2(2n+4)}{n+2} = 4 \]

Por ello, $C_{n+1} \le 4 C_n \le 4 \cdot 4^n = 4^{n+1}$. Queda así probado que
$4^n$ es una cota superior para $C_n$.

%Creo que sobra lo siguiente.
%Mediante la fórmula de Stirling podemos mostrar que:

%\[\lim_{n \to \infty} \frac{C_n}{\frac{4^n}{n^{3/2}\sqrt{\pi}}} = 1\]

%Es decir, podemos aproximar $C_n \sim 4^n$.

De esta forma, para árboles binarios de $n$ nodos, el número máximo de bits que
necesitaríamos para realizar un etiquetado unívoco sería $\lceil \log_2(C_n)
\rceil \simeq \log_2(4^n) = 2n$. Por ello, una buena solución no debería necesitar
más del doble del número de nodos de bits para guardar la estructura de un
árbol suficientemente grande. La solución propuesta se ajustará precisamente a esta
cota superior. \\

Esta aproximación no tiene en cuenta el guardado de las etiquetas de los nodos ni
el hecho de que debe diferenciarse la clave de las etiquetas dentro del archivo.

\section{Estructura de fichero}

El archivo se compondrá de la yuxtaposición de una \textbf{clave} y un
bloque de \textbf{datos}.

\begin{itemize}
	\item La \textbf{clave} será una lista de bits de longitud indefinida.
  Para cada nodo del árbol se añadirá un 1 a la lista si el nodo tiene un hijo a
  la izquierda o un 0 en caso contrario, y a continuación se añadirá un 1 si el
  nodo tiene un hijo a la derecha o un 0 en caso contrario. Este procedimiento
  se repite con los nodos hijos si existen (primero el de la izquierda y después
  el de la derecha), de forma que se almacena el par de bits de cada nodo en preorden.

  Para que la siguiente sección no comience en una posición intermedia de un byte,
  si al escribir la lista de bits en disco el número de elementos no es un múltiplo
  de 8, se añadirán ceros al final hasta que lo sea. (Esto en ningún caso
  incrementará el tamaño del archivo, puesto que, como el próximo bloque siempre
  tiene por tamaño un múltiplo de un byte, el sistema operativo añadiría bits al
  final del archivo hasta que el tamaño fuese un múltiplo de un byte).

	\item El bloque de \textbf{datos} lo formará el grupo de datos, almacenados
  en preorden sin espacios entre ellos. La posición donde empieza un nuevo dato
  se determinará a partir del tamaño de los tipos de datos (si son de tamaño fijo)
  o de la posición de un elemento terminador (si son de tamaño variable).
\end{itemize}

En principio este formato no es compatible con un posible árbol vacío, ya que
presupone que el nodo raíz existe. Esto se solventa con facilidad haciendo que se
guarden los árboles vacíos como archivos vacíos y considerando los archivos vacíos
como árboles vacíos.

\section{Procedimientos}

Tanto al leer como al escribir la clave del árbol, será necesario avanzar en
bloques de dos bits en lugar de bloques de un byte. En una implementación de estos
métodos se requerirá la simulación de entrada y salida en bloques más pequeños de
un byte leyendo o extrayendo un byte y haciendo cuatro operaciones, una para cada
grupo de dos bits, con el byte.

\subsection{Lectura}

A la hora de leer el archivo que almacena el árbol, en principio no se sabe dónde
termina la clave y dónde empieza el bloque de etiquetas. Aunque es posible
determinar esa posición leyendo el archivo para después generar simultáneamente la
estructura del árbol y su contenido, optaremos en lugar de eso por obtener en
primer lugar la estructura y después rellenar las etiquetas con los valores
correctos, lo que permitirá leer el archivo de forma lineal, sin saltos. \\

Para obtener la estructura del árbol se partirá de un árbol con un único nodo (a
no ser, claro está, que el archivo esté vacío) y se leerán los bits de dos en dos.
En cada par de bits, si el primero de ellos es un 1, se le añadirá al nodo actual
un hijo a la izquierda; y si el segundo es un 1, se le asignará un hijo a la
derecha. Tras añadir (o no) los hijos, se pasa al siguiente nodo en preorden. Este
procedimiento sigue hasta que se llegue a un nodo vacío. \\

Una vez generada la estructura, se lee el bloque de datos, asignando a cada nodo la
etiqueta encontrada en el archivo, recorriendo los nodos del árbol en preorden. A
la hora de leer la primera etiqueta, si el último par de bits leído no constituía
los dos últimos bits de un byte, se pasará al siguiente byte, puesto que el bloque
de datos empezará en el comienzo de un byte y no en una posición intermedia. \\

A continuación exponemos el algoritmo que genera la estructura. Inicialmente le
pasamos el nodo raíz (que se generará de antemano a no ser que el archivo sea
vacío, en cuyo caso este algoritmo no se ejecutará) y la posición 0. Tanto el nodo
como la posición serán tratados por referencia por el algoritmo. La posición debe
tratarse por referencia porque, hasta que no se haya generado completamente la
estructura que surge del hijo a la izquierda, no se conocerá la posición en la que
comienza la información a partir del hijo a la derecha.

\begin{algo}
\AvanzaArmazon{$n,s,p$}: \\
\KwIn{$n$, nodo actual; $s$, cadena a leer, $p$; posición en la cadena}

\BlankLine

\texttt{izq} $ := s[p] == $ `1'\;
\texttt{der} $ := s[p+1] == $ `1'\;
\BlankLine

\If{\texttt{izq}}{\textbf{Inserta} hijo izquierdo a $n$}
\If{\texttt{der}}{\textbf{Inserta} hijo derecho a $n$}

\BlankLine

$p \;+\!\!= 2$\;

\BlankLine

\If{\texttt{izq}}{\AvanzaArmazon{$n.\operatorname{izquierdo},s,p$}}
\If{\texttt{der}}{\AvanzaArmazon{$n.\operatorname{derecho},s,p$}}

\end{algo}

El bloque de datos se leería a continuación, a partir de la posición inmediatamente
posterior al último byte del archivo que correspondía a la clave.

\subsection{Escritura}

Para guardar el árbol simplemente escribiremos en un archivo la clave del árbol
seguida del bloque de datos. \\

Para cada nodo, se añadirá un 1 a la clave si tiene un hijo a la izquierda o un 0
en caso contrario, y se añadirá un 1 si tiene un hijo a la derecha o un 0 en caso
contrario. Si el hijo a la izquierda existía, se repetirá este proceso en el hijo
a la izquierda; y si el hijo a la derecha existía, se procederá igual con este hijo
a la derecha (después de terminar con el hijo a la izquierda si este existía). De
esta forma se recorren los nodos en preorden y se rellena la cadena. \\

Después se rellena el bloque de datos (saltando al siguiente byte si el último
byte en el que se escribió no quedó completo) leyendo las etiquetas en preorden. \\

Así queda el algoritmo que escribe la clave. Inicialmente lo llamaremos con el
nodo raíz y una cadena vacía (que será modificada por referencia) como parámetros.
Para simplificar su exposición, definiremos la operación \texttt{+=} entre cadenas
como la concatenación o anexión a la primera cadena, en su final, de la segunda
cadena.

\begin{algo}
\GuardaBits{$n,s$}: \\
\KwIn{$n$, nodo actual; $s$, cadena}

\BlankLine

\eIf{$n$ tiene hijo \textbf{izquierdo}}{$s \; +\!\!=$ `1'}{$s \; +\!\!=$ `0'}
\eIf{$n$ tiene hijo \textbf{derecho}}{$s \; +\!\!=$ `1'}{$s \; +\!\!=$ `0'}

\BlankLine

\If{$n$ tiene hijo \textbf{izquierdo}}{\GuardaBits{$n.\operatorname{izquierdo},s$}}
\If{$n$ tiene hijo \textbf{derecho}}{\GuardaBits{$n.\operatorname{derecho},s$}}

\end{algo}

A continuación se escribiría en el archivo el contenido de las etiquetas, en
preorden. Si el árbol resultaba ser vacío, se escribirá un archivo vacío y no se
ejecutará este algoritmo ni la escritura de etiquetas posterior.

\subsection{Lectura de información del árbol}

Es posible obtener cierta información de la estructura del árbol simplemente
leyendo la clave, sin generar el árbol completo:

\begin{itemize}
	\item \textbf{Offset:} Es posible determinar la posición donde comienza el
	bloque de datos en el archivo. Para ello, fijamos una variable entera con
	signo a 1, y por cada byte de archivo le sumamos el peso de Hamming de ese
	byte (es decir, el número de bits que toman el valor 1) y le restamos 4, hasta
	que la variable tome un valor no positivo. Cuando esto ocurra, el byte que
	debería leerse a continuación es el mismo byte donde comienza el bloque de
	datos.

	La variable representa el número de nodos que están pendientes de ser añadidos
	al árbol a cada momento. Cuando esta llega a 0, no quedan más nodos que añadir,
	por lo que lo siguiente será el bloque de datos.
	\item \textbf{Tamaño:} Puede obtenerse el número de nodos del árbol incluso
	si el tamaño del tipo de dato es variable (si fuese fijo, sería aún más fácil
	de obtener a partir del tamaño del archivo). Esto se puede consguir sumando el
	offset (según se obtuvo en el procedimiento anterior y contando el inicio de
	archivo como la posición 0) multiplicado por 4 al valor que tomó la variable al
	ejecutar el procedimiento anterior.

	Al hacer esto se está anulando la resta de 4 por cada byte al número de nodos
	pendiente de ser añadidos a cada paso. Por ello, se obtendrá el número total
	de nodos.
\end{itemize}

\section{Implementación}

Para probar la solución propuesta se adjunta una implementación de dos clases que
aplican los procedimientos descritos y un código que obtiene un árbol de gran
tamaño con etiquetas y estructura al azar, lo guarda según este procedimiento, lo
lee de nuevo y verifica que el árbol leído de disco y el que se había guardado
coinciden. El árbol generado no se borra tras la ejecución del programa, por lo
que puede explorarse con un editor hexadecimal tras la ejecución. \\

El programa puede modificarse para que el tipo de dato usado para las etiquetas o
su máximo valor posible sea distinto. En el ejemplo se ha escogido el tipo de dato
\textbf{int} y un máximo de \textbf{100} para que, al leer el archivo con el editor
hexadecimal, sea evidente dónde termina la clave inicial y dónde comienzan las
etiquetas. También puede modificarse la función que decide si se añadirá o no
un nuevo elemento, para que el número de componentes sea menor o mayor. Según está
en el ejemplo, el árbol rondará las \textbf{175 000 componentes}, pero reduciendo o
aumentando la constante $10.0$ de la función \texttt{NuevoElemento} se puede
reducir o aumentar, respectivamente, el número de componentes. \\

La implementación del tipo de dato de árbol binario es la misma que se ha puesto
como ejemplo en la plataforma de la asignatura, con cambios menores.
\end{document}
