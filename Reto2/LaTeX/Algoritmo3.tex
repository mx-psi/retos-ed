\section{Tercer algoritmo}

\subsection{Almacenamiento y presentación de resultados}

Este tercer y último algoritmo recorre por profundidad todos los caminos
posibles utilizando un vector ordenado. Su principal ventaja es su \textbf{reducido
uso de memoria}. Las estructuras de datos que utiliza el algoritmo son 3:

\begin{description}
	\item[Operaciones] Una operación es una tripleta $(a,b,\circ)$ que representa
	la operación $a \circ b$.
	\item[Pila] El algoritmo utiliza una pila de operaciones que usará, como
	máximo un total de 6 elementos. Notaremos la pila como $P$ y sus elementos (al tratarla como
	lista) como $P_i$. La pila \textbf{almacena las operaciones realizadas}.
	\item[Lista ordenada] Esta lista \textbf{ordenada} contiene
	\textbf{los elementos con los que	se opera} en cada iteración.
	Debido a las transformaciones realizadas su tamaño variará entre \textbf{2} y
	\textbf{6} elementos. La notaremos como $L$ y sus elementos como $L_i$.
\end{description}

% Falta la representación de resultados

Para la representación de resultados debemos acceder a la pila como si
fuera un vector para hacer que el algoritmo sea más eficiente.
Describimos un algoritmo recursivo que nos permite imprimir las operaciones
hasta llegar al final. Utilizamos 3 funciones auxiliares que dependen
de la implementación y que son:

\begin{itemize}
	\item $R((a,b,\circ)) = a \circ b$
	\item $\operatorname{Arg1}((a,b,\circ)) = a$
	\item $\operatorname{Arg2}((a,b,\circ)) = b$
\end{itemize}

Con estas operaciones podemos describir el algoritmo:

\begin{algo}
\Imprime{$U,n$}: \\
\KwIn{$U$, vector que nos marca los elementos usados,
$n$ posición a imprimir}

  \If{$R(P_n) \in L$ y no ha sido usado}{
		Márcalo como usado \;
		\textbf{Termina}\;
	}

	\BlankLine

	\ForEach{$i \in [0,n-1]$}{
		\If{$R(P_i) = \operatorname{Arg1}(P_i)$}{
			\Imprime{$U,i$}\;
			\textbf{Sal} del bucle\;
		}
	}

	\BlankLine

	\ForEach{$i \in [0,n-1]$}{
		\If{$R(P_i) = \operatorname{Arg2}(P_i)$}{
			\Imprime{$U,i$}\;
			\textbf{Sal} del bucle\;
		}
	}

  \textbf{Imprimir} $P_n$\;
\end{algo}

De esta forma, sólo tendremos que llamar $\texttt{Imprime}(v,n)$ con
$n$ el último índice de la pila y $v$ un vector de 6 elementos booleanos
falsos para imprimir la solución.

\subsection{Caso base}

El algoritmo es un algoritmo recursivo y presentamos de forma separada el caso
base y el caso general. El \textbf{caso base} sucede cuando $|L| = 2$
es decir, tenemos una lista de la forma $[L_0, L_1]$ con $L_0 \leq L_1$.
En tal caso recorremos todas las combinaciones y si alguna llega a la solución,
la devolvemos:

\begin{algo}
	\KwIn{$L$, lista con 2 elementos y $S$}
	\KwOut{Si se ha conseguido llegar al objetivo}
	\ForEach{$\circ \in \{+,-,\cdot,/\}$}{
		$R := L_1 \circ L_0$\;
		\If{$R$ no es válido}{\textbf{Continuar}\;}
		\BlankLine
		\If{$|R-S| < |M_a-S|$}{
		$M_a := R$\;
		\If{$res = S$}{
			\textbf{Poner} $(L_1,L_0, \circ)$ en $P$\;
			\KwRet{\texttt{true}}
		}
		}
	}
	\KwRet{\texttt{false}}
\end{algo}

La operación no será válida si no cumple las condiciones indicadas en la introducción,
es decir, el resultado es uno de los operandos o es 0.

\subsection{Caso general}

En el caso general recorremos $L$ y realizamos para cada pareja válida cada operación.
Reducimos por tanto el tamaño de $L$ a exactamente un elemento menos y llamamos
recursivamente al algoritmo:


\begin{algo}
\Cifras{$L,S,M_a,P$}: \\
\KwIn{$L$, lista con $n>2$ elemento, $S$, $M_a$, $P$}
\KwOut{Si se ha conseguido llegar al objetivo}
\ForEach{$\circ \in \{+,-,\cdot,/\}$}{
\If{$\circ$ usa 1}{$i := 0$\;}
\Else{Asignar a $i$ el primer índice tal que $L_i \neq 1$\;}
\BlankLine

\While{$i < |L|$}{
$j:=i+1$\;
\While{$j < |L|$}{
$R := L_j \circ L_i$\;
\lIf{$R$ no es válido}{\textbf{Continuar}}

\BlankLine
\textbf{Poner} $(L_j,L_i, \circ)$ en $P$\;
\textbf{Borrar} $L_i,L_j$ de $L$\;
\textbf{Insertar} $R$ en $L$\;
\BlankLine

\If{$|R-S| < |M_a-S|$}{
$M_a := R$\;
\lIf{$res = S$}{\KwRet{\texttt{true}}}
}
\BlankLine

\lIf{\Cifras{$L,S,M_a,P$}}{\KwRet{\texttt{true}}}
\BlankLine

\textbf{Quitar} de $P$\;
\textbf{Borrar} $R$ de $L$\;
\textbf{Insertar} $L_i,L_j$ en $L$\;

Avanzar hasta el siguiente $j$ con $L_j$ diferente\;
}
Avanzar hasta el siguiente $i$ con $L_i$ diferente\;
}
}
\KwRet{\texttt{false}}
\caption{Caso general del tercer algoritmo}
\end{algo}

Aprovechamos que el algoritmo está ordenado para evitar comprobar
casos duplicados y evitar productos y divisiones por 1.
La pila puede almacenar resultados basura que no contribuyan a
llegar al resultado, por lo que debemos imprimir sólo las que
necesitamos de acuerdo al algoritmo de impresión visto en la
sección anterior.

\subsection{Características del algoritmo}

Cada operación realizada sobre la lista $L$ antes de la llamada
recursiva borra dos elementos e inserta uno, lo que reduce el
tamaño de ésta en 1. Además, cuando restauramos el estado inicial
de la lista, esta no puede aumentar su tamaño. Es decir, la lista
\textbf{no puede tener nunca más de 6 elementos}, la cantidad
de elementos que tiene inicialmente.

Por otra parte, la pila $P$ sólo puede aumentar de tamaño al
reducirse el de $L$, lo que limita su tamaño también a
\textbf{un máximo de 6 elementos} (cada uno de ellos con 3
enteros). Es decir, además de las variables locales sólo
necesitamos tener en memoria un total de \textbf{24 enteros}
para ejecutar el algoritmo.

Esto nos limita a la hora de ofrecer la solución cuando no llegamos
a la mejor aproximación, ya que la pila estará vacía en este
caso. Para obtener la solución en este caso deberíamos llamar
al algoritmo de nuevo con la mejor aproximación.

La otra gran ventaja del algoritmo es que nos permite podar gran
cantidad de casos al estar $L$ ordenada. Algunas de las optimizaciones
que podemos realizar son:

\begin{itemize}
	\item Saltar elementos repetidos.
	\item Evitar repeticiones por conmutatividad y casos no válidos
	como resultados negativos o divisiones que nunca serán válidas.
	\item Evitar utilizar unos cuando la operación sea producto o suma.
	\item Podríamos, utilizando el algoritmo de búsqueda binaria,
	tratar la división como un caso especial y buscar sólo los múltiplos
	del número más pequeño.
\end{itemize}
