\section{Segundo Algoritmo}
Mientras que el algoritmo anterior construía todos los caminos empezando por los
menos profundos y los exploraba a cada vez mayor profundidad, este algoritmo
optará por explorar los caminos en profundidad. Esto tendrá la ventaja de que no
será necesario mantener en memoria todos los caminos en ningún momento, pero
tendrá el problema de que, si el resultado exacto no se encuentra, habremos
perdido el camino por el que se llegó al mismo. El procedimiento será recursivo.\\

Tomaremos $L$ como entrada. Combinamos pares de elementos de $L$ con las operaciones
según se describieron, obteniendo 60 elementos. Comprobamos si hemos llegado al objetivo:
si el elemento está entre estos elementos, lo hemos encontrado y podremos
reconstruir la secuencia de operaciones. De lo contrario, para cada elemento
obtenido (o hasta que encontremos el número), creamos un vector $\bar L$ con este
elemento y los elementos de $L$ que no se han usado para crearlo y repetimos
el proceso con $\bar L$ como entrada. \\

El algoritmo es el siguiente:

\begin{algo}
 \KwIn{$L$,$S$}
 \KwOut{Operaciones hasta llegar a $S$}
 \hspace{0.25cm}	Sean $b_0,..,b_m$ resultados de efectuar todas las operaciones posibles sobre pares $L_p$ y $L_q$ con $p < q$ \;
 \ForAll{$i$ desde $0$ hasta $m$}{
  \eIf{$b_i$ = $S$}{
   \KwRet{Devuelve los $L_j, L_k$ que se usaron para crear el $b_i$}
   }{
	Repetir el algoritmo con $\{L_j : L_j$ no fue usado para obtener $b_i\} \cup b_i$ \;
  }
 }
\end{algo}

\subsection{Características del algoritmo}

La principal característica de esta implementación es que solo tenemos en memoria
aquellos elementos necesarios para seguir avanzando. Una vez que ya hemos
recorrido todos los posibles elementos asociados al elemento $b_i$,
estos dejan de ser necesarios y se eliminan de la memoria. \\

Sin embargo, el principal problema es que, tal cual está ahora, no permite hallar
la mejor aproximación.

\subsection{Mejor Aproximación}
El problema de encontrar la mejor aproximación es bastante sencillo una vez
ya creado el algoritmo de búsqueda, basta con crear un campo nuevo en el que
vamos guardando la mejor aproximación, $M_a$, que debe ser externo a la función:
\vspace{0.25cm}

\begin{algo}
	\KwIn{$L$,$S$}
	\KwOut{Operaciones hasta llegar a $S$}
	\hspace{0.25cm}	Sean $b_0,..,b_m$ resultados de efectuar todas las operaciones posibles sobre pares $L_p$ y $L_q$ con $p < q$ \;
	\ForAll{$i$ desde $0$ hasta $m$}{
		\eIf{$b_i$ = $S$}{
			\KwRet{Devuelve los $L_j, L_k$ que se usaron para crear el $b_i$}
		}{
		\If{$|b_i-M_a|$ $<$ $|M_a-S|$}{$M_a \leftarrow b_i$\;}
		Repetir el algoritmo con $\{L_j : L_j$ no fue usado para obtener $b_i\} \cup b_i$ \;
	}
}
\end{algo}

Ahora podemos identificar el resultado aproximado más cercano. Sin embargo
sigue sin ser posible construirlo; tendremos que llamar de nuevo al algoritmo
con el resultado aproximado más cercano para poder construir las operaciones.
