\section{Combinaciones mágicas}

\begin{definition}
  Una combinación $C$ se dice \textbf{mágica} si $[100,999] \subseteq C^{\ast}$.
\end{definition}

No existen propiedades destacables de las combinaciones mágicas, que nos permitan
acotar el número de casos a probar. Proponemos una refinación de la fuerza bruta,
que compruebe todas las combinaciones una a una en busca de las que sean mágicas. \\

En primer lugar calcularemos el número de combinaciones posibles, que son 
los conjuntos $C$ posibles a escoger de $C_T$. Como el número de elementos de
$C_T$ es $|C_T| = 14$, el número de elementos de $C$ es $|C| = 6$, y
podemos repetir elementos, el número de elementos se corresponderá con
las combinaciones con repetición de 14 elementos tomados de 6 en 6:

\[\operatorname{CR}_{14}^{6} = 27\ 132 \textit{ combinaciones posibles}\]

\subsection{Versión 1}
Podría ocurrírsenos ejecutar 900 veces el algoritmo con cada combinación para
ver si encuentra todos los números desde 100 hasta 999.
Teniendo en cuenta que el tiempo de ejecución del peor caso (que se da cuando
el algoritmo no para antes de que termine el bucle, es decir, cuando el número
no se puede obtener exacto) para el algoritmo de menor tiempo de ejecución
(el algoritmo 1) en nuestra máquina ronda las tres milésimas, el
tiempo de ejecución total estaría acotado superiormente por:
$$ 0.003\ \frac{segundos}{ejecución} \cdot 27\ 132\ combinaciones \cdot 900\  \frac{ejecuciones}{combinación}=73\ 256.4\ s \approx 20\ horas$$
No es un tiempo demasiado elevado teniendo en cuenta que bastaría con obtener
las combinaciones mágicas una vez. Sin embargo, es fácil reducirlo modificando
ligeramente los algoritmos.

\subsection{Versión 2}
Hay una forma de reducir el tiempo de ejecución. Dado que cada vez que
ejecutamos el algoritmo exploramos los mismos caminos, con la única diferencia de
que se parará antes si se encuentra el objetivo, podemos modificar los
algoritmos para que no paren hasta el final y vayan marcando un número entre
100 y 999 cada vez que lo encuentren en el proceso (por ejemplo, en una
matriz de booleanos). Al final del algoritmo se comprobaría si todo número entre
100 y 999 aparece marcado, y si así es, la combinación es mágica. Esto solo
requerirá para cada combinación un tiempo similar (en realidad, ligeramente
inferior) al de comprobación de un único objetivo para la combinación que
no tenga solución exacta.

\subsubsection{Algoritmo de Verificación}
Creamos un vector con 900 booleanos (o 1000, si queremos que se marque el número
$n$ en la posición $n$ y no calcular $n-100$ a cada paso) con valor \textbf{false}.
Ejecutamos el algoritmo modificado, de manera que si un valor generado se encuentra
entre 100 y 999 esa posición toma el valor \textbf{true}. Si al final de la
ejecución todas las componentes desde 100 hasta 999 tienen el valor \textbf{true},
la combinación es mágica. 

Teniendo en cuenta que de esta manera nos quitamos el factor 900, el tiempo total,
usando 0,003 segundos como el tiempo en el peor caso, tenemos que:
$$ 0.003\ \frac{segundos}{ejecución} \cdot 27\ 132\ combinaciones \cdot 1\  \frac{ejecución}{combinación}=81.396\ segundos$$

Parece que merece la pena haberse parado a pensar antes de poner el programa a
procesar durante veinte horas. En el algoritmo 1, debido a que se reduce el número
de operaciones en el bucle más interno (no hay que calcular la mejor aproximación,
lo que evita tener que calcular dos diferencias a cada paso), el tiempo resultante
es, en nuestra máquina, aproximadamente 53 segundos. Como muestra,
en una ejecución realizada en el momento en que se redactó esto, el tiempo de
ejecución ha resultado ser de \textbf{52,1 segundos}.