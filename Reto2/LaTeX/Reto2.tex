\documentclass{article}
\usepackage{multirow}
\usepackage[spanish]{babel}
\usepackage{lmodern}
\usepackage{amssymb,amsmath,amsthm}
\usepackage[T1]{fontenc}
\usepackage[utf8]{inputenc}
\usepackage{amsfonts,dsfont}
\usepackage{enumitem}
\usepackage[margin=1in]{geometry}

% Para el código, por si queremos al menos mencionar una posible implementación
% del tipo de dato
\usepackage{listings}
\usepackage{xcolor}
\definecolor{gray}{rgb}{0.5,0.5,0.5}
\newcommand{\n}[1]{{\color{gray}#1}}
\lstset{numbers=left,numberstyle=\small\color{gray}}

\newcommand{\algorithm}[4]{
	\fbox{\parbox{14cm}{
		\textbf {\Large Algoritmo #1:} \\
		\textbf {Entrada:} #2 \\
		\textbf {Salida:} #3 \\
		\textbf {Procedimiento:} \\
		#4
	}}
}

\newcommand{\cifrasalg}[2]{
	\algorithm{#1}{$C = \{c_1, c_2, c_3, c_4, c_5, c_6\}$, $T \in [100, 999]$}{
	$L$ y $N$ tales que $d(N,T)$ es mínima y puede construirse la serie de
	operaciones con el algoritmo de obtención de operaciones}{#2}
}

\newcommand{\abs}[1]{|#1|}
\theoremstyle{definition}
\newtheorem{definition}{Definición}


\title{Reto 2}
\date{Estructura de Datos}
\author{Pablo Baeyens Fernández\\José Manuel Muñoz Fuentes\\Darío Sierra Martínez}

\begin{document}
\maketitle

\section{Introducción}
\subsection{Problema}

\begin{definition}
	$n \in \mathbb{N}$ se dice \textbf{generable por $C$} si $n \in C$ o
	existen $a,b$ generables por $C$ tal que $n = a \circ b$ para cierta operación
	$\circ \in \{+,-,/,\cdot \}$. El conjunto de los elementos generables por $C$ se nota $C^{\ast}$.
\end{definition}

El problema consiste en describir el procedimiento de un algoritmo con la
siguiente entrada y salida:

\begin{description}
	\item[Entrada:] Multiconjunto $C \subseteq C_T = [1,10] \cup \{ 25, 50, 75, 100\}$
	de 6 elementos (no necesariamente distintos) y un entero $S \in [100, 999]$ llamado
	\textbf{solución}.
	\item[Salida:] Lista de operaciones básicas ($+$,$-$,$\cdot$,$/$) tales que:
	\begin{enumerate}
		\item Cada operación usa sólo elementos generables por $C$.
		\item Cada número se utiliza sólo una vez (contando repeticiones).
		\item El último resultado es $T \in C^{\ast}$ tal que $d(S,T) = \underset{N \in C^{\ast}}{\min} d(S,N)$
	\end{enumerate}
\end{description}

\subsection{Operaciones posibles}

Podemos notar que para cumplir las condiciones de la salida del algoritmo,
dados 2 elementos $a,b$ existen cómo máximo 4 operaciones válidas:

\begin{enumerate}
	\item $a+b$. Basta considerar un orden ya que es conmutativa.
	\item $a\cdot b$. Basta considerar un orden ya que es conmutativa.
	\item $\abs{a-b}$. Esta operación se corresponde con la resta del menor al mayor.
	\item $a/b$. Sólo si $b$ divide a $a$.
\end{enumerate}


\subsection{Combinaciones posibles de elementos}

En esta sección damos cotas al tamaño de $C^{\ast}$. En primer lugar calculamos
los conjuntos $C$ posibles a escoger de $C_T$. Como $|C_T| = 14$, $|C| = 6$ y
podemos repetir elementos esto resulta en:

\[\operatorname{CR}_{14}^{6} = 27132 \textit{ combinaciones posibles}\]

Cada elemento puede combinarse una única vez con otro de $C$
y ninguno puede combinarse consigo mismo. El número total de parejas que podemos
obtener de $C$ son entonces $C_{6}^{2}=15$.
Como cada pareja puede operarse 4 veces tenemos un conjunto de 60 elementos.
Llamamos a este conjunto $C^1$.

Cada elemento de $C^1$ puede combinarse con los 4 elementos de $C$ que no se
usaron para crearlo, obteniendo un conjunto de 5 elementos;
si esto lo hacemos con todos los elementos del conjunto de las combinaciones
obtendremos todos los posibles conjuntos de 5 elementos resultantes de operar
2 de $C$.

En general, si llamamos $C^{n+1}$ al conjunto de combinaciones válidas obtenidas
a partir de $C^n$ y aplicamos el razonamiento anterior podemos acotar $C^{\ast}$
por:

\[ 4 \cdot C_6^2 \cdot 4 \cdot C_5^2 \cdot 4 \cdot C_4^2 \cdot 4 \cdot C_3^2 \cdot 4 \cdot C_2^2= 2764000\]

De esta forma, podemos definir $T:\mathbb{N} \to \mathbb{N}$:
\[T_n(i)=n^i\prod_{k=0}^{i}C_{i-k}^{2}\]

donde $T_n(|C|)$ devuelve el tamaño de los generables a partir de $C$ con $n$
operaciones. La división no siempre es válida, por lo que tenemos que:

\[T_3(|C|) < C^{\ast} < T_4(|C|)\]

\subsection{Mejor aproximación}
\subsection{El tipo de dato}

\section{Primer algoritmo}

\subsection{Almacenamiento y presentación de resultados}
Este algoritmo tiene como objetivo construir todos los caminos posibles
que pueden seguirse usando los números de $C$ y las operaciones disponibles,
dando una solución exacta en cuanto se encuentre o el camino con una solución lo
 más cercana posible a $S$. Para ello, ampliará $L$ con listas de 4 elementos
$[n, i, j, op]$, donde $n$ es un entero obtenido a partir de los elementos que ocupan
las posiciones $i$ y $j$ en $L$ mediante la operación $op$.\\

Por comodidad,
definiremos $L_i$ como el $i$-ésimo elemento de $L$, haremos referencia a los
elementos de $L_i$ con $L_i.[n/i/j/op]$, y ``el valor de $L_i$'' hará referencia
a $L_i.n$ si $L_i$ es lista o a $L_i$ si es un entero.
Esta estructura de $L$ nos permitirá obtener la secuencia de pasos que
se han seguido hasta llegar a cualquier elemento de $L$
mediante el siguiente procedimiento recursivo:\\

\begin{algorithm}[H]
	\KwIn{$L$ y $N$, con $N = [n, i, j, op]$ o $N = n$}
	\KwOut{Las operaciones con elementos de $L$ para llegar a $n$}
	\If{$N \neq n$}{
	Aplica este algoritmo con entrada $L$, $L_i$\;
	Aplica este algoritmo con entrada $L$, $L_j$\;
	\If{$op$ no es conmutativa y $L_i.n < L_j.n$}{intercambia $i$ y $j$\;}}
	\KwRet{\texttt{a operador b = n}, con $a = L_i.n$, $b = L_j.n$}
\caption{Obtención de operaciones}
\end{algorithm}

Se observa que este procedimiento conmuta correctamente los operandos en el caso
de las restas y los cocientes a la hora de mostrar la operación.

\subsection{Algoritmo}

\emph{Nota: en esta sección se definen funciones para poder describir el algoritmo.
Estas funciones no tienen por qué existir en un hipotético programa que ponga en
práctica este algoritmo. De hecho, ofrecería un mejor resultado introducir los
valores que toman estas funciones para cada $L_i$ como un elemento más en la
lista que lo constituye.}

\begin{definition}
  Sea $L_k \in L$:
  \begin{itemize}
    \item $L_k$ es de \textbf{primera generación}, $\gen(L_k) = 1$ si $L_k \in C$.
    \item $L_k$ es de \textbf{$n$-ésima generación}, $\gen(L_k) = n$ si $\gen(L_{L_k.i}) + \gen(L_{L_k.j}) = n$.
  \end{itemize}
  Los \textbf{miembros} de una generación $i$ son $\operatorname{miembros}(i) = \{L_k : gen(k) = i\}$.
\end{definition}

La generación representa el número de elementos que han sido necesarios
para obtener $L_k$, o también el número de operaciones que se han
requerido más $1$.
El algoritmo se ejecutará de forma que, en cada iteración, se obtendrán
elementos que serán exclusivamente de una generación en particular,
controlando la generación de los dos elementos de los que procede cada uno.

Los conjuntos de miembros nos permiten diferenciar los elementos de $L$ según en
qué momento de la ejecución del algoritmo han sido obtenidos y determinar el
inicio y el final de los bucles que recorren los elementos con los que se operará
en cada paso.

\begin{definition}
  Sea $L_k \in L$. Los \textbf{números usados por $L_k$}, $U(L_k)$ son:
  \begin{itemize}
    \item $U(L_k) = \{ L_k\}$ si $L_k \in C$
    \item $U(L_k) = U(L_{L_k.i}) \cup U(L_{L_k.j})$, en otro caso.
  \end{itemize}
  Dos elementos de $L$, $L_a$ y $L_b$, \textbf{se solapan} si
  $U(L_a) \cap  U(L_b) \ne \emptyset$.
\end{definition}

De esta forma, $U(L_k)$ representa el conjunto de números de $C$ que han
intervenido para obtener $L_k$, y dos elementos se solaparán cuando tengan
algún elemento inicial en común. Esto será crucial para poder comprobar a cada
paso del algoritmo si es posible operar con dos elementos entre sí: cada par
de elementos que se usen para añadir uno nuevo deben no solaparse. \\

Se almacenará en $N$ la mejor aproximación de $S$.
Cuando termine el algoritmo, se consultará $N$ y se generarán las
operaciones realizadas con el algoritmo anterior. El algoritmo que se propone
devuelve un $L$ y $N$ con los que se podrá construir, con el algoritmo anterior, una
solución que se acerque lo más posible al objetivo:\\

\begin{algorithm}[H]
\KwIn{$C = \{c_1, c_2, c_3, c_4, c_5, c_6\}$, $T \in [100, 999]$}
\KwOut{$L$ y $N$ con $d(N,T)$ mínima.}

$L = [c_1, c_2, c_3, c_4, c_5, c_6]$\;
$N = c_6$\;
\ForAll{$g = 2$ hasta $6$}{
	\ForAll{$i$ tal que $2 \cdot \gen(L_i) \le g$}{
		\ForAll{$j \in miembros(g-\gen(L_i))$ tal que $j > i$:}{
			\If{$U(L_i) \cap  U(L_j) = \emptyset$}{
				\ForAll{$op \in \{+,-,\cdot,/\}$}{
					\If{$[n, i, j, op]$, con $n = L_i.n\ op\ L_j.n$, es \textbf{válida}}{
					 Añadir $[n, i, j, op]$ a $L$\;
           \If{$|n-T| < |N.n-T|$}{
             $N.n \leftarrow [n, i, j, op]$
               \If{$N = T$}
               {\KwRet{$L,N$}}
           }
					}
				}
			}
		}
	}
}
\KwRet{$L,N$}
\caption{Obtención de la solución}
\end{algorithm}

\vspace{0.4cm}

Para definir el inicio y el final de los bucles que recorren $L$, se empleará una
lista $G$ de 6 elementos, que indicará en la posición $G_i$ el primer elemento
de $L$ que es de $i$-ésima generación. Definiremos también $G_7$ como la posición
siguiente a la del último elemento de $L$.
La función $\operatorname{miembros}(i)$ podría redefinirse aprovechando $G$ como
 $\{L_k : k \ge G_i, k < G_{i+1}\}$.

% Los dejo comentados porque el código está sujeto a cambios. Cuando esté listo del todo, lo reescribiré (sin borrar el actual) con esto en los bucles:
% Los dos bucles quedarían así ('hasta' indica el último elemento, no el primero que no está):
%Para $i = 1$ hasta $G_{((g+1)/2+1)}-1$:\\
%Para $j = \max \{i+1, G(g-gen(L_i))\}$ hasta $G(g-gen(L_i)+1)\}$:\\

\subsection{Características del algoritmo}
Este algoritmo explora todas las posibles operaciones válidas.
Por ello, garantiza encontrar la solución más cercana posible a la pedida.
Además, dado que construyen las generaciones una a una y en orden ascendente,
se garantiza que el número de operaciones usadas para obtener un valor que se
distancie lo mínimo posible a la solución será el mínimo, tanto en el caso de
resultado exacto como en caso de valor aproximado.\\

El algoritmo no ordena los operandos en cada resta o cociente aunque sea necesario.
En cualquier caso, como siempre hay un único resultado válido, el algoritmo que
presenta la solución se encargará de mostrar los operandos en el orden correcto.

\section{Combinaciones mágicas}

\begin{definition}
  Una combinación $C$ se dice \textbf{mágica} si $[100,999] \subseteq C^{\ast}$.
\end{definition}
\subsection{Algoritmo para buscar combinaciones mágicas}


\end{document}
