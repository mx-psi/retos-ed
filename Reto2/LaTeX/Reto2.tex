\documentclass{article}
\usepackage{multirow}
\usepackage[spanish]{babel}
\usepackage{lmodern}
\usepackage{amssymb,amsmath,amsthm}
\usepackage[spanish,onelanguage]{algorithm2e}
\usepackage[T1]{fontenc}
\usepackage[utf8]{inputenc}
\usepackage{amsfonts,dsfont}
\usepackage{enumitem}
\usepackage[margin=1in]{geometry}

% Para el código, por si queremos al menos mencionar una posible implementación
% del tipo de dato
\usepackage{listings}
\usepackage{xcolor}
\definecolor{gray}{rgb}{0.5,0.5,0.5}
\newcommand{\n}[1]{{\color{gray}#1}}
\lstset{numbers=left,numberstyle=\small\color{gray}}

\newcommand{\abs}[1]{|#1|}
\theoremstyle{definition}
\newtheorem{definition}{Definición}


\title{Reto 2}
\date{Estructura de Datos}
\author{Pablo Baeyens Fernández\\José Manuel Muñoz Fuentes\\Darío Sierra Martínez}

\begin{document}
\maketitle
\section{Introducción}
\subsection{Problema}

\begin{definition}
	$n \in \mathbb{N}$ se dice \textbf{generable por $C$} si $n \in C$ o
	existen $a,b$ generables por $C$ (generadores de $n$) tales que:
	\begin{itemize}
		\item $n = a \circ b$ para cierta operación $\circ \in \{+,-,/,\cdot \}$.
		\item Si $a,b \in C$, $a \neq b$ contando repetición y en otro caso sus
		generadores son distintos.
	\end{itemize}
	El conjunto de los elementos generables por $C$ se nota $C^{\ast}$.
\end{definition}

El problema consiste en describir el procedimiento de un algoritmo con la
siguiente entrada y salida:

\begin{description}
	\item[Entrada:] La entrada consiste de:
		\begin{itemize}
			\item Multiconjunto $C \subseteq C_T = [1,10] \cup \{ 25, 50, 75, 100\}$
			de 6 elementos (no necesariamente distintos).
			\item Entero $S \in [100, 999]$ llamado \textbf{solución}.
		\end{itemize}

	\item[Salida:] Lista de operaciones básicas ($+$,$-$,$\cdot$,$/$) tales que:
	\begin{enumerate}
		\item Cada operación usa sólo elementos generables por $C$.
		\item Cada número se utiliza sólo una vez (contando repeticiones).
		\item El último resultado es $T \in C^{\ast}$ tal que $d(S,T) = \underset{N \in C^{\ast}}{\min} d(S,N)$
	\end{enumerate}
\end{description}

Dado que utilizamos sólo estructuras de datos lineales, llamaremos $L$ a la lista
que se identifica con el multiconjunto $C$.

\subsection{Operaciones posibles}

Podemos notar que para cumplir las condiciones de la salida del algoritmo,
dados 2 elementos $a,b$ existen cómo máximo 4 operaciones válidas:

\begin{enumerate}
	\item $a+b$. Basta considerar un orden ya que es conmutativa.
	\item $a\cdot b$. Basta considerar un orden ya que es conmutativa.
	\item $\abs{a-b}$. Esta operación se corresponde con la resta del menor al mayor.
	\item $a/b$. Sólo si $b$ divide a $a$.
\end{enumerate}

Sea $a \circ b = c$ una cierta operación.
Podemos considerar algunos criterios para reducr el número de operaciones:

\begin{itemize}
	\item Si $c = a$ o $c = b$ la operación puede omitirse de cualquier solución.
	Esto sólo ocurre cuando uno de los operandos es $1$.
	\item Si $c = 0$ la operación puede omitirse de cualquier solución, ya que de
	un 0 solo puede obtenerse otro 0 o un valor ya existente. Podemos redefinir
	la división como 0 cuando no es válida y entrará en ese caso.
	\item Si $a$ surgió de la misma operación que se está realizando
  ($d \circ e = a, (d \circ e) \circ b = c$),
  puede omitirse porque:
	\begin{itemize}
		\item Si $\circ$ es conmutativa ($+$ o $\cdot$)
		$(d \circ e) \circ b = c$ equivale a $d \circ (e \circ b) = c$.
		Podemos hacer sólo un caso.
		\item Si $\circ$ no es conmutativa ($-$ o $/$) y es válida
		$(d \circ e) \circ b = c$ equivale a $d \circ (e \circ^{-1} b) = c$
		siendo $\circ^{-1}$ la operación opuesta. Podemos hacer sólo un caso.
	\end{itemize}
	\item Si $b$ surgió a partir de la misma operación $\circ$ que se está realizando
	($d \circ e = b, a \circ (d \circ e) = c$),
	en algunos casos también es posible descartar la operación; a saber:
	\begin{itemize}
		\item Si $\circ$ no es conmutativa y es válida, la expresión
		$a \circ (d \circ e)\ = c$ equivale a $(a \circ^{-1} e) \circ d\ = c$.
		Podemos hacer sólo un caso.
		\item Si $\circ$ es conmutativa, y la posición de $d$ se ha obtenido antes
		que $a$, $a \circ (d \circ e)\ = c$	equivale a $d \circ (a \circ e) = c$.
		Podemos hacer sólo un caso.
	\end{itemize}
\end{itemize}

Decimos que una operación es \textbf{válida} si no puede omitirse por lo
expuesto anteriormente.

\subsection{Combinaciones posibles de elementos}

En esta sección damos cotas al tamaño de $C^{\ast}$. En primer lugar calculamos
los conjuntos $C$ posibles a escoger de $C_T$. Como $|C_T| = 14$, $|C| = 6$ y
podemos repetir elementos esto resulta en:

\[\operatorname{CR}_{14}^{6} = 27132 \textit{ combinaciones posibles}\]

Cada elemento puede combinarse una única vez con otro de $C$
y ninguno puede combinarse consigo mismo. El número total de parejas que podemos
obtener de $C$ son entonces $C_{6}^{2}=15$.
Como cada pareja puede operarse 4 veces tenemos un conjunto de 60 elementos.
Llamamos a este conjunto $C^1$.

Cada elemento de $C^1$ puede combinarse con los 4 elementos de $C$ que no se
usaron para crearlo, obteniendo un conjunto de 5 elementos;
si esto lo hacemos con todos los elementos del conjunto de las combinaciones
obtendremos todos los posibles conjuntos de 5 elementos resultantes de operar
2 de $C$.

En general, si llamamos $C^{n+1}$ al conjunto de combinaciones válidas obtenidas
a partir de $C^n$ y aplicamos el razonamiento anterior podemos acotar $C^{\ast}$
por:

\[ 4 \cdot C_6^2 \cdot 4 \cdot C_5^2 \cdot 4 \cdot C_4^2 \cdot 4 \cdot C_3^2 \cdot 4 \cdot C_2^2= 2764000\]

De esta forma, podemos definir $T:\mathbb{N} \to \mathbb{N}$:
\[T_n(i)=n^i\prod_{k=0}^{i}C_{i-k}^{2}\]

donde $T_n(|C|)$ devuelve el tamaño de los generables a partir de $C$ con $n$
operaciones. La división no siempre es válida, por lo que tenemos que:

\[T_3(|C|) < C^{\ast} < T_4(|C|)\]

\section{Primer algoritmo}

\subsection{Almacenamiento y presentación de resultados}
Este algoritmo tiene como objetivo construir todos los caminos posibles
que pueden seguirse usando los números de $C$ y las operaciones disponibles,
dando una solución exacta en cuanto se encuentre o el camino con una solución lo
más cercana posible a $S$. Para ello, ampliará $L$ (que era la lista que se
identificaba con $C$) con listas de 4 elementos $[n, i, j, op]$, donde $n$ es un
entero obtenido a partir de los elementos que ocupan las posiciones $i$ y $j$ en
$L$ mediante la operación $op$, según se definieron las operaciones en la
introducción.\\

Por comodidad,
definiremos $L_i$ como el $i$-ésimo elemento de $L$, haremos referencia a los
elementos de $L_i$ con $L_i.[n/i/j/op]$, y ``el valor de $L_i$'' hará referencia
a $L_i.n$ si $L_i$ es lista o a $L_i$ si es un entero.
Esta estructura de $L$ nos permitirá obtener la secuencia de pasos que
se han seguido hasta llegar a cualquier elemento de $L$
mediante el siguiente procedimiento recursivo:\\

\begin{algorithm}[H]
	\KwIn{$L$ y $N$, con $N = [n, i, j, op]$ o $N = n$}
	\KwOut{Las operaciones con elementos de $L$ para llegar a $n$}
	\If{$N \neq n$}{
	Aplica este algoritmo con entrada $L$, $L_i$\;
	Aplica este algoritmo con entrada $L$, $L_j$\;
	\If{$op$ no es conmutativa y $L_i.n < L_j.n$}{intercambia $i$ y $j$\;}
	\KwRet{\texttt{a operador b = n}, con $a = L_i.n$, $b = L_j.n$}
	}
\caption{Obtención de operaciones}
\end{algorithm}

\vspace{0.4cm}

Se observa que este procedimiento conmuta correctamente los operandos en el caso
de las restas y los cocientes a la hora de mostrar la operación, por lo que el
algoritmo no tendrá que preocuparse por almacenar el orden en el que se encontraban
los operandos (aunque sí por efectuar correctamente cada operación).

\subsection{Algoritmo}

\emph{Nota: en esta sección se definen funciones para poder describir el algoritmo.
Estas funciones no tienen por qué existir en un hipotético programa que ponga en
práctica este algoritmo. De hecho, ofrecería un mejor resultado introducir los
valores que toman estas funciones para cada $L_i$ como un elemento más en la
lista que lo constituye.}

\begin{definition}
  Sea $L_k \in L$:
  \begin{itemize}
    \item $L_k$ es de \textbf{primera generación}, $\gen(L_k) = 1$ si $L_k \in C$
    (es decir, si es uno de los 6 elementos iniciales).
    \item $L_k$ es de \textbf{$n$-ésima generación}, $\gen(L_k) = n$ si $\gen(L_{L_k.i}) + \gen(L_{L_k.j}) = n$.
  \end{itemize}
  Los \textbf{miembros} de una generación $i$ son $\operatorname{miembros}(i) = \{L_k : gen(k) = i\}$.
\end{definition}

La generación representa el número de elementos que han sido necesarios
para obtener $L_k$, o también el número de operaciones que se han
requerido más $1$.
El algoritmo se ejecutará de forma que, en cada iteración, se obtendrán
elementos que serán exclusivamente de una generación en particular,
controlando la generación de los dos elementos de los que procede cada uno.\\

Los conjuntos de miembros nos permiten diferenciar los elementos de $L$ según en
qué momento de la ejecución del algoritmo han sido obtenidos y determinar el
inicio y el final de los bucles que recorren los elementos con los que se operará
en cada paso. Como el algoritmo generará los elementos de generación en generación,
será posible determinar la generación de un elemento por su posición.

\begin{definition}
  Sea $L_k \in L$. Los \textbf{números usados por $L_k$}, $U(L_k)$ son:
  \begin{itemize}
    \item $U(L_k) = \{ L_k\}$ si $L_k \in C$
    \item $U(L_k) = U(L_{L_k.i}) \cup U(L_{L_k.j})$, en otro caso.
  \end{itemize}
  Dos elementos de $L$, $L_a$ y $L_b$, \textbf{se solapan} si
  $U(L_a) \cap  U(L_b) \ne \emptyset$.
\end{definition}

De esta forma, $U(L_k)$ representa el conjunto de números de $C$ que han
intervenido para obtener $L_k$, y dos elementos se solaparán cuando tengan
algún elemento de $C$ en común. Esto será crucial para poder comprobar a cada
paso del algoritmo si es posible operar con dos elementos entre sí: cada par
de elementos que se usen para añadir uno nuevo deben no solaparse. \\

Se almacenará en $N$ la mejor aproximación de $S$. Cuando termine el
algoritmo, se consultará $N$ y se generarán las operaciones realizadas para
obtener $N$ con el algoritmo anterior.\\

Finalmente, el algoritmo queda así:\\

\begin{algorithm}[H]
\KwIn{$C = \{c_1, c_2, c_3, c_4, c_5, c_6\}$, $S \in [100, 999]$}
\KwOut{$L$ y $N$ con $|N-S|$ mínima.}

$L = [c_1, c_2, c_3, c_4, c_5, c_6]$\;
$N = c_6$\;
\ForAll{$g$ desde $2$ hasta $6$}{
	\ForAll{$i$ tal que $2 \cdot \gen(L_i) \le g$}{
		\ForAll{$j \in miembros(g-\gen(L_i))$ tal que $j > i$}{
			\If{$U(L_i) \cap  U(L_j) = \emptyset$}{
				\ForAll{$op \in \{+,-,\cdot,/\}$}{
					\If{$[n, i, j, op]$, con $n = L_i.n\ op\ L_j.n$, es \textbf{válida}}{
					 Añadir $[n, i, j, op]$ a $L$\;
           \If{$|n-S| < |N.n-S|$}{
             $N.n \leftarrow [n, i, j, op]$\;
               \If{$N.n = S$}
               {\KwRet{$L,N$}}
           }
					}
				}
			}
		}
	}
}
\KwRet{$L,N$}
\caption{Obtención de la solución}
\end{algorithm}

\vspace{0.4cm}

Puede observarse que, cuando $g = 6$, si el nuevo $n$ no está más cerca de $S$ que $N.n$, no es necesario añadir $[n,i,j,op]$ a $L$. Añadir esta condición oscurece el algoritmo, pero se tendrá en cuenta en la implementación.\\

Para definir el inicio y el final de los bucles que recorren $L$, se empleará una
lista $G$ de 6 elementos, que indicará en la posición $G_i$ el primer elemento
de $L$ que es de $i$-ésima generación. Definiremos también $G_7$ como la posición
siguiente a la del último elemento de $L$.
La función $\operatorname{miembros}(i)$ podría redefinirse aprovechando $G$ como
$\{L_k : k \ge G_i, k < G_{i+1}\}$. \\

El bucle que recorre los $i$ debe terminar al final de la última generación $\bar g$ tal que $2\cdot \bar g \le g$, que es la posición anterior a la que comienza la generación posterior. \\

El bucle que recorre los $j$ debe recorrer la generación que, sumada a la de $i$, dé $g$; pero si ambas fuesen la misma, debería empezar después de $i$. \\

Con los bucles escritos de esta forma, el algoritmo queda así:

\begin{algorithm}[H]
	\KwIn{$C = \{c_1, c_2, c_3, c_4, c_5, c_6\}$, $S \in [100, 999]$}
	\KwOut{$L$ y $N$ con $|N-S|$ mínima.}
	
	$L = [c_1, c_2, c_3, c_4, c_5, c_6]$\;
	$N = c_6$\;
	\ForAll{$g$ desde $2$ hasta $6$}{
		\ForAll{$i = 1$ hasta $G_{((g+1)/2+1)}-1$}{
			\ForAll{$j = \max \{i+1, G(g-gen(L_i))\}$ hasta $G(g-gen(L_i)+1)\}$}{
				\If{$U(L_i) \cap  U(L_j) = \emptyset$}{
					\ForAll{$op \in \{+,-,\cdot,/\}$}{
						\If{$[n, i, j, op]$, con $n = L_i.n\ op\ L_j.n$, es \textbf{válida}}{
							Añadir $[n, i, j, op]$ a $L$\;
							\If{$|n-S| < |N.n-S|$}{
								$N.n \leftarrow [n, i, j, op]$\;
								\If{$N.n = S$}
								{\KwRet{$L,N$}}
							}
						}
					}
				}
			}
		}
	}
	\KwRet{$L,N$}
	\caption{Obtención de la solución (bucles for)}
\end{algorithm}

\subsection{Características del algoritmo}
Este algoritmo explora todas las posibles operaciones válidas.
Por ello, garantiza encontrar la solución más cercana posible a la pedida.
Además, dado que construyen las generaciones una a una y en orden ascendente,
se garantiza que el número de operaciones usadas para obtener un valor que se
distancie lo menos posible a la solución será el mínimo, tanto en el caso de
resultado exacto como en caso de valor aproximado.\\

Este algoritmo destaca por el bajo tiempo de ejecución de su implementación. Usando la versión modificada para hallar combinaciones mágicas de este algoritmo, se han obtenido todas las combinaciones mágicas en $66,46$ segundos.\\

El algoritmo no ordena los operandos en cada resta o cociente aunque sea necesario.
En cualquier caso, como siempre hay un único resultado válido, el algoritmo que
presenta la solución se encargará de mostrar los operandos en el orden correcto.

\section{Combinaciones mágicas}

\begin{definition}
  Una combinación $C$ se dice \textbf{mágica} si $[100,999] \subseteq C^{\ast}$.
\end{definition}

No existen propiedades destacables de las combinaciones mágicas, que nos permitan
acotar el número de casos a probar. Proponemos una refinación de la fuerza bruta:

\subsection{Versión 1}
Podría ocurrírsenos ejecutar 900 veces el algoritmo con cada combinación para
ver si encuentra todos los números del 100, y eso ejecutarlo con las 27 mil
combinaciones.

Teniendo en cuenta que el tiempo de ejecución del peor caso (cualquier número
que no se pueda obtener), en nuestra máquina ronda la décima de segundo el % Depende del algoritmo
tiempo de ejecución total vendría dado por:
$$ 0.1\frac{segundos}{ejecucion} \cdot 27'132 combinaciones  \cdot 900 \frac{ejecuciones}{combinacion}=2'441'880s \simeq 1 mes$$
Evidentemente esto no es viable.

\subsection{Versión 2}
Si nos damos cuenta de que no tenemos que ejecutar el algoritmo 900 veces la
cosa mejora, si buscamos un número que de antemano sabemos que no puede estar
(ej: -1), estaremos generando todos los números posibles con el vector de
entrada, de manera que podemos saber con una única ejecuión si la
combinación es o no mágica.

\subsubsection{Algoritmo de Verificación}
Para verificar si la combinación es mágica modificaremos el algoritmo de
busqueda anteriormente descrito. Puesto que no buscamos ningún número en
concreto modificaremos la condición de parada.

Creamos un vector con 900 booleanos, ejecutamos el algoritmo de búsqueda
modificado, de manera que si un valor generado se encuentra entre 100 y 1000
esa posición se pone a \textbf{true}. Si el vector entero está puesto a
\textbf{true} antes de la finalización del algoritmo la combinación es mágica.

Teniendo en cuenta que de esta manera nos quitamos el factor 900 el tiempo total
usando 0.1 s como el tiempo en el peor caso tenemos que:
$$ 0.1\frac{segundos}{ejecucion} \cdot 27'132 combinaciones  \cdot 1 \frac{ejecuciones}{combinacion}=2'713.2s \simeq 45 \textit{ minutos}$$
Un tiempo mucho más razonable.

\vspace{0.25cm}
Describimos aquí el algoritmo con más detalle:
\vspace{0.25cm}

\begin{algo}

 \KwIn{$L$,$S$,$M_a$,$p_0,..,p_k$ booleanos}
 \KwOut{\textbf{true}, si $p_i = \texttt{true} \; \forall i$, \texttt{false} en otro caso}
 \hspace{0.25cm}	Sean los $b_0,..,b_{m}$ resultado de combinar los $L_i$ y $L_j$ $i \neq j$  \;
 \While{$i<m$ y $\exists$ j tal que $p_j=\texttt{false}$}{

  \eIf{$b_i \geq 100$ y $b_i \leq 1000$ }{
   		$p_i=\texttt{true}$ ;
   }{

	Repetir el algoritmo con los $L_j$ que no se usaron para crear el $b_i$ y el $b_i$ \;
	Avanzar el índice $i$ \;
  }
 }
\end{algo}
Si todos los $p_k$ están puestos a $\textbf{true}$ la combinación es mágica.

\section{Tipos de datos}
\subsection{Tipo de dato para $L_k$}

Cada $L_k$ debe contener, como mínimo, tipos de datos que se identifiquen con la
 descripción que se dio anteriormente ($[n, i, j, op]$). Se implementaría
 probablemente con una estructura que permitiese distintos tipos de datos.\\

$n$ sería un tipo de dato entero con signo de, al menos, 32 bits. Es importante
que el tamaño sea de al menos 32 bits, dado que si fuese de 16 bits sería posible
 salirse del rango y obtener resultados falsos o perder resultados correctos.
\begin{itemize}
	\item Un ejemplo de resultado falso: si trabajamos con enteros de 16 bits con
	signo, $100^2 \cdot 75$ daría como resultado $29104$, que podría dar lugar a
	una solución falsa. Con $C=\{4, 6, 8, 75, 100, 100\},\ S = 911$ (ejemplo de
	problema para el que no existe una solución exacta) se obtendría esta falsa
	solución: $100 \cdot 100=10000; 10000 \cdot 75=29104;29104/8=3638;3638+6=3644;3644/4=911$.
	\item Un ejemplo de problema que no podría resolverse sería
	$C=\{3, 3, 25, 50, 75, 100\},\ S = 996$. Toda solución a este ejemplo pasa por
	obtener el número $99600$, que está fuera del rango de los enteros de 16 bits
	con o sin signo (si se ejecuta el algoritmo haciendo que se rechace todo
	resultado temporal igual a $99600$ de forma similar a como se hace con el
	$0$, no se obtiene solución exacta). Una solución (podría ser la única) es:
	$3 \cdot 50 = 53; 25 \cdot 53 = 1325; 3+1325=1328; 75 \cdot 1328 = 99600; 99600/100=996$.
\end{itemize}

$i$ y $j$ serían dos enteros de 32 bits que indicarían la posición de los $L_i$
y $L_j$ de los que se obtuvo el nodo anterior. Dado que los $L_k$ deben tener
todos la misma estructura, y en lo propuesto los seis primeros elementos (que
procedían de $C$) estaban compuestos solo por un número, para esos seis
elementos iniciales podrían tomar un valor arbitrario (por ejemplo, el valor
inválido -1, lo que requeriría enteros con signo). \\

$op$ representaría la operación con la que se obtuvo el elemento.
Podría representarse con cualquier tipo de dato. Para los seis primeros
elementos, podría tomar un valor arbitrario, preferiblemente inválido.\\

De cara al algoritmo que genera las operaciones con las que se llegó a un
elemento, interesaría que, en los seis primeros elementos, al menos uno de los
tres últimos elementos que componen esta estructura tuviese un valor inválido,
dado que el algoritmo debe no hacer nada cuando está siendo ejecutado con uno
de los seis primeros elementos como entrada. \\

Además de los datos anteriores, que son indispensables, probablemente interesará
 añadir más elementos a la estructura:

\begin{itemize}
	\item Incluir un entero que indique la generación del elemento podría reducir
	el tiempo de ejecución del algoritmo respecto del tiempo que necesitaría si
	emplease una función que calculase la generación de un elemento explorando los
	 elementos de los que procede. Un entero de 16 bits, que tomase el valor de la
	  suma de las generaciones de los elementos de los que procede y que tomase el
		 valor 1 para los elementos iniciales, sería suficiente para ello.
	\item Sería particularmente interesante incluir un entero sin signo de al
	menos 16 bits que indique en el $i$-ésimo bit menos significativo si se ha
	usado el $i$-ésimo elemento del multiconjunto de números disponibles $C$. Tal
	bit sería $1$ si lo ha usado, o $0$ si no. Esto supondría que la comprobación
	de si dos elementos se solapan sería tan simple como comprobar si la operación
	lógica $AND$ de ambos es un valor no nulo. Al añadir nuevos elementos a $L$,
	este dato tomaría el valor de la operación lógica $OR$ sobre los datos en los
	dos elementos de los que se obtiene el nuevo. Por ejemplo, si el valor de
	este dato para dos elementos fuese $001011$ y $110000$, el elemento que se
	obtendría al operar con ambos tendría en este dato el valor $111011$. Este
	dato tomaría, para el elemento inicial $i$-ésimo, el valor $2^{i-1}$ o
	$1 << i$. ($1$ para el primero, $2$ para el segundo, $4$ para el tercero \ldots)
\end{itemize}

\subsection{Tipo de dato para $L$}

\end{document}
