\section{Tipos de datos}
\subsection{Tipo de dato para $L_k$}

Cada $L_k$ debe contener, como mínimo, tipos de datos que se identifiquen con la
 descripción que se dio anteriormente ($[n, i, j, op]$). Se implementaría
 probablemente con una estructura que permitiese distintos tipos de datos.\\

$n$ sería un tipo de dato entero con signo de, al menos, 32 bits. Es importante
que el tamaño sea de al menos 32 bits, dado que si fuese de 16 bits sería posible
 salirse del rango y obtener resultados falsos o perder resultados correctos.
\begin{itemize}
	\item Un ejemplo de resultado falso: si trabajamos con enteros de 16 bits con
	signo, $100^2 \cdot 75$ daría como resultado $29104$, que podría dar lugar a
	una solución falsa. Con $C=\{4, 6, 8, 75, 100, 100\},\ S = 911$ (ejemplo de
	problema para el que no existe una solución exacta) se obtendría esta falsa
	solución: $100 \cdot 100=10000; 10000 \cdot 75=29104;29104/8=3638;3638+6=3644;3644/4=911$.
	\item Un ejemplo de problema que no podría resolverse sería
	$C=\{3, 3, 25, 50, 75, 100\},\ S = 996$. Toda solución a este ejemplo pasa por
	obtener el número $99600$, que está fuera del rango de los enteros de 16 bits
	con o sin signo (si se ejecuta el algoritmo haciendo que se rechace todo
	resultado temporal igual a $99600$ de forma similar a como se hace con el
	$0$, no se obtiene solución exacta). Una solución (podría ser la única) es:
	$3 \cdot 50 = 53; 25 \cdot 53 = 1325; 3+1325=1328; 75 \cdot 1328 = 99600; 99600/100=996$.
\end{itemize}

$i$ y $j$ serían dos enteros de 32 bits que indicarían la posición de los $L_i$
y $L_j$ de los que se obtuvo el nodo anterior. Dado que los $L_k$ deben tener
todos la misma estructura, y en lo propuesto los seis primeros elementos (que
procedían de $C$) estaban compuestos solo por un número, para esos seis
elementos iniciales podrían tomar un valor arbitrario (por ejemplo, el valor
inválido -1, lo que requeriría enteros con signo). \\

$op$ representaría la operación con la que se obtuvo el elemento.
Podría representarse con cualquier tipo de dato. Para los seis primeros
elementos, podría tomar un valor arbitrario, preferiblemente inválido.\\

De cara al algoritmo que genera las operaciones con las que se llegó a un
elemento, interesaría que, en los seis primeros elementos, al menos uno de los
tres últimos elementos que componen esta estructura tuviese un valor inválido,
dado que el algoritmo debe no hacer nada cuando está siendo ejecutado con uno
de los seis primeros elementos como entrada. \\

Además de los datos anteriores, que son indispensables, probablemente interesará
 añadir más elementos a la estructura:

\begin{itemize}
	\item Incluir un entero que indique la generación del elemento podría reducir
	el tiempo de ejecución del algoritmo respecto del tiempo que necesitaría si
	emplease una función que calculase la generación de un elemento explorando los
	 elementos de los que procede. Un entero de 16 bits, que tomase el valor de la
	  suma de las generaciones de los elementos de los que procede y que tomase el
		 valor 1 para los elementos iniciales, sería suficiente para ello.
	\item Sería particularmente interesante incluir un entero sin signo de al
	menos 16 bits que indique en el $i$-ésimo bit menos significativo si se ha
	usado el $i$-ésimo elemento del multiconjunto de números disponibles $C$. Tal
	bit sería $1$ si lo ha usado, o $0$ si no. Esto supondría que la comprobación
	de si dos elementos se solapan sería tan simple como comprobar si la operación
	lógica $AND$ de ambos es un valor no nulo. Al añadir nuevos elementos a $L$,
	este dato tomaría el valor de la operación lógica $OR$ sobre los datos en los
	dos elementos de los que se obtiene el nuevo. Por ejemplo, si el valor de
	este dato para dos elementos fuese $001011$ y $110000$, el elemento que se
	obtendría al operar con ambos tendría en este dato el valor $111011$. Este
	dato tomaría, para el elemento inicial $i$-ésimo, el valor $2^{i-1}$ o
	$1 << i$. ($1$ para el primero, $2$ para el segundo, $4$ para el tercero\ldots)
\end{itemize}

\subsection{Tipo de dato para $L$}
