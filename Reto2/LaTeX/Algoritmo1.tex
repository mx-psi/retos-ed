\section{Primer algoritmo}

\subsection{Almacenamiento y presentación de resultados}
Este algoritmo tiene como objetivo construir todos los caminos posibles
que pueden seguirse usando los números de $C$ y las operaciones disponibles,
dando una solución exacta en cuanto se encuentre o el camino con una solución lo
más cercana posible a $S$. Para ello, ampliará $L$ (que era la lista que se
identificaba con $C$) con listas de 4 elementos $[n, i, j, op]$, donde $n$ es un
entero obtenido a partir de los elementos que ocupan las posiciones $i$ y $j$ en
$L$ mediante la operación $op$, según se definieron las operaciones en la
introducción.\\

Por comodidad,
definiremos $L_i$ como el $i$-ésimo elemento de $L$, haremos referencia a los
elementos de $L_i$ con $L_i.[n/i/j/op]$, y ``el valor de $L_i$'' hará referencia
a $L_i.n$ si $L_i$ es lista o a $L_i$ si es un entero.
Esta estructura de $L$ nos permitirá obtener la secuencia de pasos que
se han seguido hasta llegar a cualquier elemento de $L$
mediante el siguiente procedimiento recursivo:\\

\begin{algorithm}[H]
	\KwIn{$L$ y $N$, con $N = [n, i, j, op]$ o $N = n$}
	\KwOut{Las operaciones con elementos de $L$ para llegar a $n$}
	\If{$N \neq n$}{
	Aplica este algoritmo con entrada $L$, $L_i$\;
	Aplica este algoritmo con entrada $L$, $L_j$\;
	\If{$op$ no es conmutativa y $L_i.n < L_j.n$}{intercambia $i$ y $j$\;}
	\KwRet{\texttt{a operador b = n}, con $a = L_i.n$, $b = L_j.n$}
	}
\caption{Obtención de operaciones}
\end{algorithm}

\vspace{0.4cm}

Se observa que este procedimiento conmuta correctamente los operandos en el caso
de las restas y los cocientes a la hora de mostrar la operación, por lo que el
algoritmo no tendrá que preocuparse por almacenar el orden en el que se encontraban
los operandos (aunque sí por efectuar correctamente cada operación).

\subsection{Algoritmo}

\emph{Nota: en esta sección se definen funciones para poder describir el algoritmo.
Estas funciones no tienen por qué existir en un hipotético programa que ponga en
práctica este algoritmo. De hecho, ofrecería un mejor resultado introducir los
valores que toman estas funciones para cada $L_i$ como un elemento más en la
lista que lo constituye.}

\begin{definition}
  Sea $L_k \in L$:
  \begin{itemize}
    \item $L_k$ es de \textbf{primera generación}, $\gen(L_k) = 1$ si $L_k \in C$
    (es decir, si es uno de los 6 elementos iniciales).
    \item $L_k$ es de \textbf{$n$-ésima generación}, $\gen(L_k) = n$ si $\gen(L_{L_k.i}) + \gen(L_{L_k.j}) = n$.
  \end{itemize}
  Los \textbf{miembros} de una generación $i$ son $\operatorname{miembros}(i) = \{L_k : gen(k) = i\}$.
\end{definition}

La generación representa el número de elementos que han sido necesarios
para obtener $L_k$, o también el número de operaciones que se han
requerido más $1$.
El algoritmo se ejecutará de forma que, en cada iteración, se obtendrán
elementos que serán exclusivamente de una generación en particular,
controlando la generación de los dos elementos de los que procede cada uno.\\

Los conjuntos de miembros nos permiten diferenciar los elementos de $L$ según en
qué momento de la ejecución del algoritmo han sido obtenidos y determinar el
inicio y el final de los bucles que recorren los elementos con los que se operará
en cada paso. Como el algoritmo generará los elementos de generación en generación,
será posible determinar la generación de un elemento por su posición.

\begin{definition}
  Sea $L_k \in L$. Los \textbf{números usados por $L_k$}, $U(L_k)$ son:
  \begin{itemize}
    \item $U(L_k) = \{ L_k\}$ si $L_k \in C$
    \item $U(L_k) = U(L_{L_k.i}) \cup U(L_{L_k.j})$, en otro caso.
  \end{itemize}
  Dos elementos de $L$, $L_a$ y $L_b$, \textbf{se solapan} si
  $U(L_a) \cap  U(L_b) \ne \emptyset$.
\end{definition}

De esta forma, $U(L_k)$ representa el conjunto de números de $C$ que han
intervenido para obtener $L_k$, y dos elementos se solaparán cuando tengan
algún elemento de $C$ en común. Esto será crucial para poder comprobar a cada
paso del algoritmo si es posible operar con dos elementos entre sí: cada par
de elementos que se usen para añadir uno nuevo deben no solaparse. \\

Se almacenará en $N$ la mejor aproximación de $S$. Cuando termine el
algoritmo, se consultará $N$ y se generarán las operaciones realizadas para
obtener $N$ con el algoritmo anterior.\\

Finalmente, el algoritmo queda así:\\

\begin{algorithm}[H]
\KwIn{$C = \{c_1, c_2, c_3, c_4, c_5, c_6\}$, $S \in [100, 999]$}
\KwOut{$L$ y $N$ con $|N-S|$ mínima.}

$L = [c_1, c_2, c_3, c_4, c_5, c_6]$\;
$N = c_6$\;
\ForAll{$g$ desde $2$ hasta $6$}{
	\ForAll{$i$ tal que $2 \cdot \gen(L_i) \le g$}{
		\ForAll{$j \in miembros(g-\gen(L_i))$ tal que $j > i$}{
			\If{$U(L_i) \cap  U(L_j) = \emptyset$}{
				\ForAll{$op \in \{+,-,\cdot,/\}$}{
					\If{$[n, i, j, op]$, con $n = L_i.n\ op\ L_j.n$, es \textbf{válida}}{
					 Añadir $[n, i, j, op]$ a $L$\;
           \If{$|n-S| < |N.n-S|$}{
             $N.n \leftarrow [n, i, j, op]$\;
               \If{$N.n = S$}
               {\KwRet{$L,N$}}
           }
					}
				}
			}
		}
	}
}
\KwRet{$L,N$}
\caption{Obtención de la solución}
\end{algorithm}

\vspace{0.4cm}

Puede observarse que, cuando $g = 6$, si el nuevo $n$ no está más cerca de $S$ que $N.n$, no es necesario añadir $[n,i,j,op]$ a $L$. Añadir esta condición oscurece el algoritmo, pero se tendrá en cuenta en la implementación.\\

Para definir el inicio y el final de los bucles que recorren $L$, se empleará una
lista $G$ de 6 elementos, que indicará en la posición $G_i$ el primer elemento
de $L$ que es de $i$-ésima generación. Definiremos también $G_7$ como la posición
siguiente a la del último elemento de $L$.
La función $\operatorname{miembros}(i)$ podría redefinirse aprovechando $G$ como
$\{L_k : k \ge G_i, k < G_{i+1}\}$. \\

El bucle que recorre los $i$ debe terminar al final de la última generación $\bar g$ tal que $2\cdot \bar g \le g$, que es la posición anterior a la que comienza la generación posterior. \\

El bucle que recorre los $j$ debe recorrer la generación que, sumada a la de $i$, dé $g$; pero si ambas fuesen la misma, debería empezar después de $i$. \\

Con los bucles escritos de esta forma, el algoritmo queda así:

\begin{algorithm}[H]
	\KwIn{$C = \{c_1, c_2, c_3, c_4, c_5, c_6\}$, $S \in [100, 999]$}
	\KwOut{$L$ y $N$ con $|N-S|$ mínima.}
	
	$L = [c_1, c_2, c_3, c_4, c_5, c_6]$\;
	$N = c_6$\;
	\ForAll{$g$ desde $2$ hasta $6$}{
		\ForAll{$i = 1$ hasta $G_{((g+1)/2+1)}-1$}{
			\ForAll{$j = \max \{i+1, G(g-gen(L_i))\}$ hasta $G(g-gen(L_i)+1)\}$}{
				\If{$U(L_i) \cap  U(L_j) = \emptyset$}{
					\ForAll{$op \in \{+,-,\cdot,/\}$}{
						\If{$[n, i, j, op]$, con $n = L_i.n\ op\ L_j.n$, es \textbf{válida}}{
							Añadir $[n, i, j, op]$ a $L$\;
							\If{$|n-S| < |N.n-S|$}{
								$N.n \leftarrow [n, i, j, op]$\;
								\If{$N.n = S$}
								{\KwRet{$L,N$}}
							}
						}
					}
				}
			}
		}
	}
	\KwRet{$L,N$}
	\caption{Obtención de la solución (bucles for)}
\end{algorithm}

\subsection{Características del algoritmo}
Este algoritmo explora todas las posibles operaciones válidas.
Por ello, garantiza encontrar la solución más cercana posible a la pedida.
Además, dado que construyen las generaciones una a una y en orden ascendente,
se garantiza que el número de operaciones usadas para obtener un valor que se
distancie lo menos posible a la solución será el mínimo, tanto en el caso de
resultado exacto como en caso de valor aproximado.\\

Este algoritmo destaca por el bajo tiempo de ejecución de su implementación. Usando la versión modificada para hallar combinaciones mágicas de este algoritmo, se han obtenido todas las combinaciones mágicas en $52,1$ segundos.\\

El algoritmo no ordena los operandos en cada resta o cociente aunque sea necesario.
En cualquier caso, como siempre hay un único resultado válido, el algoritmo que
presenta la solución se encargará de mostrar los operandos en el orden correcto.

\subsection{Implementación}

Se usarán dos tipos de datos \texttt{Nodo} y \texttt{VectorNodos} que representarán a cada $L_k$ y a $L$ respectivamente.

\subsubsection{Tipo de dato para $L_k$}

Cada $L_k$ debe estar representado por una estructura que incluya, como
mínimo, tipos de datos que se identifiquen con la descripción que se dio
anteriormente ($[n, i, j, op]$).\\

$n$ será un tipo de dato entero con signo de 32 bits. Es importante
que el tamaño sea de al menos 32 bits, dado que si fuese de 16 bits sería posible
salirse del rango y obtener resultados falsos o perder resultados correctos.
\begin{itemize}
	\item Un ejemplo de resultado falso: si trabajamos con enteros de 16 bits con
	signo, $100^2 \cdot 75$ daría como resultado $29104$, que podría dar lugar a
	una solución falsa. Con $C=\{4, 6, 8, 75, 100, 100\},\ S = 911$ (ejemplo de
	problema para el que no existe una solución exacta) se obtendría esta falsa
	solución: $100 \cdot 100=10000; 10000 \cdot 75=29104;29104/8=3638;3638+6=3644;3644/4=911$.
	\item Un ejemplo de problema que no podría resolverse sería
	$C=\{3, 3, 25, 50, 75, 100\},\ S = 996$. Toda solución a este ejemplo pasa por
	obtener el número $99600$, que está fuera del rango de los enteros de 16 bits
	con o sin signo (si se ejecuta el algoritmo haciendo que se rechace todo
	resultado temporal igual a $99600$ de forma similar a como se hace con el
	$0$, no se obtiene solución exacta). Una solución (podría ser la única) es:
	$3 \cdot 50 = 53; 25 \cdot 53 = 1325; 3+1325=1328; 75 \cdot 1328 = 99600; 99600/100=996$.
\end{itemize}

$i$ y $j$ serán dos enteros de 32 bits que indicarán la posición de los $L_i$
y $L_j$ de los que se obtuvo el nodo anterior (como se verá en la siguiente sección, el tamaño de $L$ es menor que $2^{31}$). Dado que los $L_k$ deben tener
todos la misma estructura, y en lo propuesto los seis primeros elementos (que
procedían de $C$) estaban compuestos solo por un número, para esos seis
elementos iniciales podrían tomar un valor arbitrario (por ejemplo, el valor
inválido -1, lo que requeriría enteros con signo), aunque en la
implementación no se optará por ello. \\

$op$ representará la operación con la que se obtuvo el elemento.
Podría representarse con cualquier tipo de dato; optaremos por un entero
de 16 bits. Para los seis primeros elementos, podría tomar un valor
arbitrario, preferiblemente inválido.\\

De cara al algoritmo que genera las operaciones con las que se llegó a un
elemento, interesará que, en los seis primeros elementos, al menos uno de los
tres últimos elementos que componen esta estructura tuviese un valor inválido,
dado que el algoritmo debe no hacer nada cuando está siendo ejecutado con uno
de los seis primeros elementos como entrada. En la implementación optaremos
por darle un valor inválido al campo $op$. Esto facilitará la labor a la función
que comprueba si uno de los operandos procede de la misma operación. \\

Además de los datos anteriores, que son indispensables, probablemente
interesará hacer (y en la implementación se ha hecho) lo siguiente:

\begin{itemize}
	\item Incluir un entero que indique la generación del elemento podría reducir
	el tiempo de ejecución del algoritmo respecto del tiempo que necesitaría si
	emplease una función que calculase la generación de un elemento explorando los
	 elementos de los que procede. Un entero de 16 bits que tomase el valor de la
	  suma de las generaciones de los elementos de los que procede y que tomase el
		 valor 1 para los elementos iniciales sería suficiente para ello.
	\item Sería particularmente interesante incluir un entero sin signo de al
	menos 16 bits que indique en el $i$-ésimo bit menos significativo si se ha
	usado el $i$-ésimo elemento del multiconjunto de números disponibles $C$. Tal
	bit sería $1$ si lo ha usado, o $0$ si no. Esto supondría que la comprobación
	de si dos elementos se solapan sería tan simple como comprobar si la operación
	lógica $AND$ de ambos es un valor no nulo. Al añadir nuevos elementos a $L$,
	este dato tomaría el valor de la operación lógica $OR$ sobre los datos en los
	dos elementos de los que se obtiene el nuevo. Por ejemplo, si el valor de
	este dato para dos elementos fuese $001011$ y $110000$, el elemento que se
	obtendría al operar con ambos tendría en este dato el valor $111011$. Este
	dato tomaría, para el elemento inicial $i$-ésimo, el valor $2^{i-1}$ o
	$1 << i$. ($1$ para el primero, $2$ para el segundo, $4$ para el tercero\ldots)
\end{itemize}

\subsubsection{Tipo de dato para $L$}
$L$ debe estar representado por una clase que permita, al menos, añadir elementos $L_k$. Dado que el número de elementos es demasiado como para almacenarlos en la pila, se optará por una estructura lineal situada en memoria dinámica. El tipo de dato reservará espacio para un número suficiente (según se calculará a continuación) de elementos del tipo de dato descrito anteriormente, y además almacenará las posiciones en las que se inicia cada generación de elementos, para controlar dónde deben comenzar y acabar los bucles que recorren los elementos. \\

Convendrá que esta clase disponga de métodos para añadir elementos, consultar los elementos actuales, obtener la posición de inicio de cada generación y marcar como final de la generación el último elemento actual, además de un destructor que libere la memoria reservada. Para las combinaciones mágicas, se modificará ligeramente esta estructura. \\

\subsubsection{Tamaño de $L$}

El número de elementos de $L$ para los que debe reservarse memoria viene determinado por la suma de elementos de cada generación:
\begin{itemize}
	\item Para la primera generación tenemos seis elementos: los seis números iniciales.
	\item La segunda generación estará conformada por los resultados de operar con pares de elementos de la primera generación.
	\item La tercera reunirá resultados de operar con un elemento de la primera y otro de la segunda.
	\item La cuarta estará constituida tanto por operaciones con un elemento de la primera y otro de la tercera como por operaciones con pares de elementos de la segunda.
	\item La quinta tendrá elementos que procederán de otros previos de las generaciones 1 y 4 o 2 y 3.
	\item La sexta surgirá al operar elementos de las generaciones 1 y 5, 2 y 4 o 3 y 3. Pueden descartarse los elementos de esta generación que no se acerquen al objetivo más que otros elementos previos.
\end{itemize}

En general, los elementos de la generación $n$, para $n>1$, proceden de pares de elementos de las generaciones $i,j$ con $i \le n/2$ tales que $i+j=n$.\\

Si decidimos no añadir los elementos de la sexta generación a no ser que tomen un valor más cercano al objetivo $S$ que todos los demás, en esta generación habrá menos de $994$ elementos, dado que con cualquier combinación de $5$ números iniciales es posible llegar a, al menos, el valor $6$ ($(1+1+1)*(1+1)$), y aun en tal caso es imposible que haya $994$ elementos que, uno detrás de otro, sean mejores que el anterior: si hubiese $993$ seguidos, el último sería ya la solución para cualquier solución entre 100 y 999.\\

Sea $T(i)$ el número de elementos que tendremos de la generación $i$-ésima. El número de espacios que deberán reservarse en memoria será $T = \displaystyle \sum_{i=1}^6 T(i)$, o $T = \displaystyle \sum_{i=1}^5 T(i) + 993$ si excluimos la sexta generación. Ahora calcularemos una cota superior para $T$ suponiendo que todas las operaciones son válidas (es decir, que podremos hacer para cada pareja que no se solape las cuatro operaciones).\\

A la hora de operar con elementos de dos generaciones $i,j$ no necesariamente distintas, para cada elemento de la $i$-ésima generación el número de elementos que puede obtenerse de él será igual al número de elementos de $j$-ésima generación que pueden combinarse con tal elemento de $i$ (sea esto $P(i,j)$), 4 veces (una por operación, dado que suponemos que todas son válidas), es decir, $4 \cdot T(i) \cdot P(i,j)$.\\

Sea $L_i$ un elemento de $i$-ésima generación. Si entendemos los elementos usados por $L_i$ como una lista de elementos $1$ o $0$ que indican si el número inicial que se encuentra en esa posición ha sido usado o no (es prácticamente lo que se hace en la estructura de $L_k$ usando los bits de un entero como booleanos), tal lista tendrá $i$ unos y $6-i$ ceros, dado que $L_i$ habrá usado $i$ números iniciales. Para poder combinarse con $L_i$, los elementos de $j$-ésima generación $L_j$ deberán cumplir que los números usados por $L_i$ no lo estén. Por tanto, los $j$ números usados por $L_j$ estarían repartidos por los no usados por $L_i$. Dado que hay $C_{n}^k$ formas de seleccionar $k$ elementos de un grupo de $n$, y puesto que en cada generación el número de elementos cuya lista de elementos sea una en particular es el mismo para cualquier lista válida, la cantidad de elementos de $j$-ésima generación disponibles para combinar con $L_i$ será $\displaystyle P(i,j) = T(j) \cdot \frac {C_{6-i}^j} {C_6^j}$.\\

Por ello, si las generaciones $i$ y $j$ son distintas, el número de elementos que pueden obtenerse a partir de elementos de ambas es $\displaystyle 4 \cdot T(i) \cdot P(i,j) = 4 \cdot T(i) \cdot T(j) \cdot \frac {C_{6-i}^j} {C_6^j}$. Si $i=j$, tal número se reduce a la mitad, puesto que solo hay una forma de combinar cada par de números. Por comodidad definiremos la función $I(i,j) = 1/2\ si\ i=j,\ 1\ si\ i \ne j$. La función que determina cuántos elementos hay en una generación (a la que llamamos previamente $T(i)$) queda así:

$$T(1) = 6;\ T(n) = \sum_{i=1}^{n/2} 4 \cdot T(i) \cdot T(n-i) \cdot \frac {C_{6-i}^{n-i}} {C_6^{n-i}} \cdot I(i,n-i) \ \ \forall n \in \{2,3,4,5,6\}$$

Ya podemos obtener el tamaño máximo. Si incluye toda la sexta generación, $\displaystyle T = \sum_{i=1}^6 T(i) = 1\ 144\ 386$, o si no, $\displaystyle T = \sum_{i=1}^5 T(i) + 993 = 177\ 699$. Podremos fijar el tamaño de la estructura que representa a $L$ con ese valor (o cualquier valor mayor, pero eso supondría un desperdicio de memoria). \\

Cabe observar que podría acotarse más el tamaño si se tienen en cuenta las dos últimas condiciones que debía cumplir una operación para ser válida (que el primer elemento no proceda de la misma operación y que el segundo elemento no proceda de la misma operación en ciertos casos). Sin embargo, dado que en la implementación no se observa diferencia en el rendimiento si se reserva memoria para $10T$ espacios en lugar de $T$, aceptaremos $T$.
