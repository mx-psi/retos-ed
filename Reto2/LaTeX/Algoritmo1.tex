\section{Primer algoritmo}

\subsection{Almacenamiento y presentación de resultados}
Este algoritmo tiene como objetivo construir todos los caminos posibles
que pueden seguirse usando los números de $C$ y las operaciones disponibles,
dando una solución exacta en cuanto se encuentre o el camino con una solución lo
más cercana posible a $S$. Para ello, ampliará $L$ (que era la lista que se
identificaba con $C$) con listas de 4 elementos $[n, i, j, op]$, donde $n$ es un
entero obtenido a partir de los elementos que ocupan las posiciones $i$ y $j$ en
$L$ mediante la operación $op$, según se definieron las operaciones en la
introducción.\\

Por comodidad,
definiremos $L_i$ como el $i$-ésimo elemento de $L$, haremos referencia a los
elementos de $L_i$ con $L_i.[n/i/j/op]$, y ``el valor de $L_i$'' hará referencia
a $L_i.n$ si $L_i$ es lista o a $L_i$ si es un entero.
Esta estructura de $L$ nos permitirá obtener la secuencia de pasos que
se han seguido hasta llegar a cualquier elemento de $L$
mediante el siguiente procedimiento recursivo:\\

\begin{algorithm}[H]
	\KwIn{$L$ y $N$, con $N = [n, i, j, op]$ o $N = n$}
	\KwOut{Las operaciones con elementos de $L$ para llegar a $n$}
	\If{$N \neq n$}{
	Aplica este algoritmo con entrada $L$, $L_i$\;
	Aplica este algoritmo con entrada $L$, $L_j$\;
	\If{$op$ no es conmutativa y $L_i.n < L_j.n$}{intercambia $i$ y $j$\;}
	\KwRet{\texttt{a operador b = n}, con $a = L_i.n$, $b = L_j.n$}
	}
\caption{Obtención de operaciones}
\end{algorithm}

\vspace{0.4cm}

Se observa que este procedimiento conmuta correctamente los operandos en el caso
de las restas y los cocientes a la hora de mostrar la operación, por lo que el
algoritmo no tendrá que preocuparse por almacenar el orden en el que se encontraban
los operandos (aunque sí por efectuar correctamente cada operación).

\subsection{Algoritmo}

\emph{Nota: en esta sección se definen funciones para poder describir el algoritmo.
Estas funciones no tienen por qué existir en un hipotético programa que ponga en
práctica este algoritmo. De hecho, ofrecería un mejor resultado introducir los
valores que toman estas funciones para cada $L_i$ como un elemento más en la
lista que lo constituye.}

\begin{definition}
  Sea $L_k \in L$:
  \begin{itemize}
    \item $L_k$ es de \textbf{primera generación}, $\gen(L_k) = 1$ si $L_k \in C$
    (es decir, si es uno de los 6 elementos iniciales).
    \item $L_k$ es de \textbf{$n$-ésima generación}, $\gen(L_k) = n$ si $\gen(L_{L_k.i}) + \gen(L_{L_k.j}) = n$.
  \end{itemize}
  Los \textbf{miembros} de una generación $i$ son $\operatorname{miembros}(i) = \{L_k : gen(k) = i\}$.
\end{definition}

La generación representa el número de elementos que han sido necesarios
para obtener $L_k$, o también el número de operaciones que se han
requerido más $1$.
El algoritmo se ejecutará de forma que, en cada iteración, se obtendrán
elementos que serán exclusivamente de una generación en particular,
controlando la generación de los dos elementos de los que procede cada uno.\\

Los conjuntos de miembros nos permiten diferenciar los elementos de $L$ según en
qué momento de la ejecución del algoritmo han sido obtenidos y determinar el
inicio y el final de los bucles que recorren los elementos con los que se operará
en cada paso. Como el algoritmo generará los elementos de generación en generación,
será posible determinar la generación de un elemento por su posición.

\begin{definition}
  Sea $L_k \in L$. Los \textbf{números usados por $L_k$}, $U(L_k)$ son:
  \begin{itemize}
    \item $U(L_k) = \{ L_k\}$ si $L_k \in C$
    \item $U(L_k) = U(L_{L_k.i}) \cup U(L_{L_k.j})$, en otro caso.
  \end{itemize}
  Dos elementos de $L$, $L_a$ y $L_b$, \textbf{se solapan} si
  $U(L_a) \cap  U(L_b) \ne \emptyset$.
\end{definition}

De esta forma, $U(L_k)$ representa el conjunto de números de $C$ que han
intervenido para obtener $L_k$, y dos elementos se solaparán cuando tengan
algún elemento de $C$ en común. Esto será crucial para poder comprobar a cada
paso del algoritmo si es posible operar con dos elementos entre sí: cada par
de elementos que se usen para añadir uno nuevo deben no solaparse. \\

Se almacenará en $N$ la mejor aproximación de $S$. Cuando termine el
algoritmo, se consultará $N$ y se generarán las operaciones realizadas para
obtener $N$ con el algoritmo anterior.\\

Finalmente, el algoritmo queda así:\\

\begin{algorithm}[H]
\KwIn{$C = \{c_1, c_2, c_3, c_4, c_5, c_6\}$, $S \in [100, 999]$}
\KwOut{$L$ y $N$ con $|N-S|$ mínima.}

$L = [c_1, c_2, c_3, c_4, c_5, c_6]$\;
$N = c_6$\;
\ForAll{$g$ desde $2$ hasta $6$}{
	\ForAll{$i$ tal que $2 \cdot \gen(L_i) \le g$}{
		\ForAll{$j \in miembros(g-\gen(L_i))$ tal que $j > i$}{
			\If{$U(L_i) \cap  U(L_j) = \emptyset$}{
				\ForAll{$op \in \{+,-,\cdot,/\}$}{
					\If{$[n, i, j, op]$, con $n = L_i.n\ op\ L_j.n$, es \textbf{válida}}{
					 Añadir $[n, i, j, op]$ a $L$\;
           \If{$|n-S| < |N.n-S|$}{
             $N.n \leftarrow [n, i, j, op]$\;
               \If{$N.n = S$}
               {\KwRet{$L,N$}}
           }
					}
				}
			}
		}
	}
}
\KwRet{$L,N$}
\caption{Obtención de la solución}
\end{algorithm}

\vspace{0.4cm}

Puede observarse que, cuando $g = 6$, si el nuevo $n$ no está más cerca de $S$ que $N.n$, no es necesario añadir $[n,i,j,op]$ a $L$. Añadir esta condición oscurece el algoritmo, pero se tendrá en cuenta en la implementación.\\

Para definir el inicio y el final de los bucles que recorren $L$, se empleará una
lista $G$ de 6 elementos, que indicará en la posición $G_i$ el primer elemento
de $L$ que es de $i$-ésima generación. Definiremos también $G_7$ como la posición
siguiente a la del último elemento de $L$.
La función $\operatorname{miembros}(i)$ podría redefinirse aprovechando $G$ como
$\{L_k : k \ge G_i, k < G_{i+1}\}$. \\

El bucle que recorre los $i$ debe terminar al final de la última generación $\bar g$ tal que $2\cdot \bar g \le g$, que es la posición anterior a la que comienza la generación posterior. \\

El bucle que recorre los $j$ debe recorrer la generación que, sumada a la de $i$, dé $g$; pero si ambas fuesen la misma, debería empezar después de $i$. \\

Con los bucles escritos de esta forma, el algoritmo queda así:

\begin{algorithm}[H]
	\KwIn{$C = \{c_1, c_2, c_3, c_4, c_5, c_6\}$, $S \in [100, 999]$}
	\KwOut{$L$ y $N$ con $|N-S|$ mínima.}
	
	$L = [c_1, c_2, c_3, c_4, c_5, c_6]$\;
	$N = c_6$\;
	\ForAll{$g$ desde $2$ hasta $6$}{
		\ForAll{$i = 1$ hasta $G_{((g+1)/2+1)}-1$}{
			\ForAll{$j = \max \{i+1, G(g-gen(L_i))\}$ hasta $G(g-gen(L_i)+1)\}$}{
				\If{$U(L_i) \cap  U(L_j) = \emptyset$}{
					\ForAll{$op \in \{+,-,\cdot,/\}$}{
						\If{$[n, i, j, op]$, con $n = L_i.n\ op\ L_j.n$, es \textbf{válida}}{
							Añadir $[n, i, j, op]$ a $L$\;
							\If{$|n-S| < |N.n-S|$}{
								$N.n \leftarrow [n, i, j, op]$\;
								\If{$N.n = S$}
								{\KwRet{$L,N$}}
							}
						}
					}
				}
			}
		}
	}
	\KwRet{$L,N$}
	\caption{Obtención de la solución (bucles for)}
\end{algorithm}

\subsection{Características del algoritmo}
Este algoritmo explora todas las posibles operaciones válidas.
Por ello, garantiza encontrar la solución más cercana posible a la pedida.
Además, dado que construyen las generaciones una a una y en orden ascendente,
se garantiza que el número de operaciones usadas para obtener un valor que se
distancie lo menos posible a la solución será el mínimo, tanto en el caso de
resultado exacto como en caso de valor aproximado.\\

Este algoritmo destaca por el bajo tiempo de ejecución de su implementación. Usando la versión modificada para hallar combinaciones mágicas de este algoritmo, se han obtenido todas las combinaciones mágicas en $66,46$ segundos.\\

El algoritmo no ordena los operandos en cada resta o cociente aunque sea necesario.
En cualquier caso, como siempre hay un único resultado válido, el algoritmo que
presenta la solución se encargará de mostrar los operandos en el orden correcto.
