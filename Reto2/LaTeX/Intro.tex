\section{Introducción}
\subsection{Problema}

\begin{definition}
	Sea $C$ un multiconjunto. Se dice que $n \in \mathbb{N}$ es \textbf{generable por $C$} si $n \in C$ o
	existen $a,b$ generables por $C$ (generadores de $n$) tales que:
	\begin{itemize}
		\item $n = a \circ b$ para cierta operación $\circ \in \{+,-,/,\cdot \}$.
		\item Si $a,b \in C$, $a \neq b$ contando repetición y en otro caso sus
		generadores son distintos.
	\end{itemize}
	El conjunto de los elementos generables por $C$ lo notaremos $C^{\ast}$.
\end{definition}

El problema consiste en describir el procedimiento de un algoritmo con la
siguiente entrada y salida:

\begin{description}
	\item[Entrada:] La entrada consiste de:
		\begin{itemize}
			\item Multiconjunto $C$ de 6 elementos de $C_T = ([1,10] \cap \mathbb N) \cup \{ 25, 50, 75, 100\}$
			no necesariamente distintos.
			\item Entero $S \in [100, 999]$ llamado \textbf{solución}.
		\end{itemize}

	\item[Salida:] Lista de operaciones binarias básicas (suma $+$, resta $-$, producto $\cdot$, cociente $/$) tales que:
	\begin{enumerate}
		\item Cada operación usa sólo elementos generables por $C$ que o bien estaban en $C$ o bien se han obtenido a través de una operación anterior en la lista.
		\item Cada número se utiliza como máximo una vez (contando repeticiones).
		\item El último resultado es $T \in C^{\ast}$ tal que $|S-T| = \displaystyle \min_{N \in C^{\ast}} |S-N|$
		\item Todo número utilizado en las operaciones es un entero no negativo.
	\end{enumerate}
\end{description}

Así, el algoritmo deberá llegar a un número lo más cercano posible a $S$
efectuando operaciones con elementos de $C$, pudiendo usar un número
solo una vez por cada aparición del mismo y pudiendo emplear números
obtenidos mediante operaciones previas. \\

Dado que solo utilizamos estructuras de datos lineales, llamaremos $L$ a la lista
que se identifica con el multiconjunto $C$.

\subsection{Operaciones posibles}

Podemos notar que para cumplir las condiciones de la salida del algoritmo,
dados 2 elementos $a,b \in \mathbb{N}^{\ast}$ existen cómo máximo 4 operaciones válidas:

\begin{enumerate}
	\item $a+b$. Basta considerar un orden ya que es conmutativa.
	\item $a\cdot b$. De nuevo, basta considerar un orden ya que es conmutativa.
	\item $\abs{a-b}$. Esta operación consiste en la resta del menor al mayor. No es necesario considerar la otra resta porque $a < b \implies a-b < 0$ y $a = b \implies a-b = b-a = 0$ (en este caso ninguna resta es posible).
	\item $\max\{a,b\}/\min\{a,b\}$. Esta operación es el cociente del mayor entre el menor, y solo es posible si el menor divide al mayor y el menor no es igual a $0$. Solo hay como máximo un orden posible porque $a < b \implies b \nmid a$ y $a = b \implies a/b = b/a$.
\end{enumerate}

Sea $a \circ b = c$ una cierta operación.
Podemos considerar algunos criterios para reducir el número de operaciones:

\begin{itemize}
	\item Si $c = a$ o $c = b$ la operación puede omitirse de cualquier solución,
	dado que habrá otra solución que omite este paso y simplemente usa $a$ o $b$, respectivamente, en lugar de $c$.
	\item Si $c = 0$ la operación puede omitirse de cualquier solución, ya que de
	un 0 solo puede obtenerse otro 0 o un valor ya existente. Podemos redefinir
	la división como 0 cuando no es válida y entrará en ese caso.
	\item Si $a$ surgió de la misma operación que se está realizando
  ($d \circ e = a, (d \circ e) \circ b = c$),
  puede omitirse porque:
	\begin{itemize}
		\item Si $\circ$ es conmutativa ($+$ o $\cdot$) y por tanto asociativa,
		$(d \circ e) \circ b = c$ equivale a $d \circ (e \circ b) = c$.
		Podemos hacer solo este segundo caso y omitir el primero.
		\item Si $\circ$ no es conmutativa ($-$ o $/$) y es válida,
		$(d \circ e) \circ b = c$ equivale a $d \circ (e \circ^{-1} b) = c$
		siendo $\circ^{-1}$ la operación opuesta. Podemos hacer solo este segundo caso y omitir el primero.
	\end{itemize}
	\item Si $b$ surgió a partir de la misma operación $\circ$ que se está realizando
	($d \circ e = b, a \circ (d \circ e) = c$),
	en algunos casos también es posible descartar la operación; a saber:
	\begin{itemize}
		\item Si $\circ$ no es conmutativa y es válida, la expresión
		$a \circ (d \circ e)\ = c$ equivale a $(a \circ^{-1} e) \circ d = c$.
		Podemos hacer solo este segundo caso y omitir el primero.
		\item Si $\circ$ es conmutativa y la posición de $d$ en $L$ es anterior a la de $a$, $a \circ (d \circ e) = c$	equivale a $d \circ (a \circ e) = c$.
		Podemos hacer solo este segundo caso y omitir el primero.
	\end{itemize}
\end{itemize}

Decimos que una operación es \textbf{válida} si no puede omitirse por lo
expuesto anteriormente.