\section{Introducción}
\subsection{Problema}

\begin{definition}
	$n \in \mathbb{N}$ se dice \textbf{generable por $C$} si $n \in C$ o
	existen $a,b$ generables por $C$ (generadores de $n$) tales que:
	\begin{itemize}
		\item $n = a \circ b$ para cierta operación $\circ \in \{+,-,/,\cdot \}$.
		\item Si $a,b \in C$, $a \neq b$ contando repetición y en otro caso sus
		generadores son distintos.
	\end{itemize}
	El conjunto de los elementos generables por $C$ se nota $C^{\ast}$.
\end{definition}

El problema consiste en describir el procedimiento de un algoritmo con la
siguiente entrada y salida:

\begin{description}
	\item[Entrada:] La entrada consiste de:
		\begin{itemize}
			\item Multiconjunto $C \subseteq C_T = [1,10] \cup \{ 25, 50, 75, 100\}$
			de 6 elementos (no necesariamente distintos).
			\item Entero $S \in [100, 999]$ llamado \textbf{solución}.
		\end{itemize}

	\item[Salida:] Lista de operaciones básicas ($+$,$-$,$\cdot$,$/$) tales que:
	\begin{enumerate}
		\item Cada operación usa sólo elementos generables por $C$.
		\item Cada número se utiliza sólo una vez (contando repeticiones).
		\item El último resultado es $T \in C^{\ast}$ tal que $d(S,T) = \underset{N \in C^{\ast}}{\min} d(S,N)$
	\end{enumerate}
\end{description}

Dado que utilizamos sólo estructuras de datos lineales, llamaremos $L$ a la lista
que se identifica con el multiconjunto $C$.

\subsection{Operaciones posibles}

Podemos notar que para cumplir las condiciones de la salida del algoritmo,
dados 2 elementos $a,b \in \mathbb{N}^{\ast}$ existen cómo máximo 4 operaciones válidas:

\begin{enumerate}
	\item $a+b$. Basta considerar un orden ya que es conmutativa.
	\item $a\cdot b$. Basta considerar un orden ya que es conmutativa.
	\item $\abs{a-b}$. Esta operación se corresponde con la resta del menor al mayor.
	\item $a/b$. Sólo si $b$ divide a $a$. Basta considerar un orden ya que si $a > b$,
	$b \nmid a$ y si $a = b$, $a/b = b/a$.
\end{enumerate}

Sea $a \circ b = c$ una cierta operación.
Podemos considerar algunos criterios para reducr el número de operaciones:

\begin{itemize}
	\item Si $c = a$ o $c = b$ la operación puede omitirse de cualquier solución,
	dado que habrá otra solución que omite este paso.
	\item Si $c = 0$ la operación puede omitirse de cualquier solución, ya que de
	un 0 solo puede obtenerse otro 0 o un valor ya existente. Podemos redefinir
	la división como 0 cuando no es válida y entrará en ese caso.
	\item Si $a$ surgió de la misma operación que se está realizando
  ($d \circ e = a, (d \circ e) \circ b = c$),
  puede omitirse porque:
	\begin{itemize}
		\item Si $\circ$ es conmutativa ($+$ o $\cdot$) y por tanto asociativa,
		$(d \circ e) \circ b = c$ equivale a $d \circ (e \circ b) = c$.
		Podemos hacer solo este segundo caso y omitir el primero.
		\item Si $\circ$ no es conmutativa ($-$ o $/$) y es válida
		$(d \circ e) \circ b = c$ equivale a $d \circ (e \circ^{-1} b) = c$
		siendo $\circ^{-1}$ la operación opuesta. Podemos hacer solo este segundo caso y omitir el primero.
	\end{itemize}
	\item Si $b$ surgió a partir de la misma operación $\circ$ que se está realizando
	($d \circ e = b, a \circ (d \circ e) = c$),
	en algunos casos también es posible descartar la operación; a saber:
	\begin{itemize}
		\item Si $\circ$ no es conmutativa y es válida, la expresión
		$a \circ (d \circ e)\ = c$ equivale a $(a \circ^{-1} e) \circ d = c$.
		Podemos hacer solo este segundo caso y omitir el primero.
		\item Si $\circ$ es conmutativa, y la posición de $d$ en $L$ es anterior a la de $a$, $a \circ (d \circ e) = c$	equivale a $d \circ (a \circ e) = c$.
		Podemos hacer solo este segundo caso y omitir el primero.
	\end{itemize}
\end{itemize}

Decimos que una operación es \textbf{válida} si no puede omitirse por lo
expuesto anteriormente.

\subsection{Combinaciones posibles de elementos}

En esta sección damos cotas al tamaño de $C^{\ast}$. En primer lugar calculamos
los conjuntos $C$ posibles a escoger de $C_T$. Como $|C_T| = 14$, $|C| = 6$ y
podemos repetir elementos esto resulta en:

\[\operatorname{CR}_{14}^{6} = 27132 \textit{ combinaciones posibles}\]

Cada elemento puede combinarse una única vez con otro de $C$
y ninguno puede combinarse consigo mismo. El número total de parejas que podemos
obtener de $C$ son entonces $C_{6}^{2}=15$.
Como cada pareja puede operarse 4 veces tenemos un conjunto de 60 elementos.
Llamamos a este conjunto $C^1$.

Cada elemento de $C^1$ puede combinarse con los 4 elementos de $C$ que no se
usaron para crearlo, obteniendo un conjunto de 5 elementos;
si esto lo hacemos con todos los elementos del conjunto de las combinaciones
obtendremos todos los posibles conjuntos de 5 elementos resultantes de operar
2 de $C$.

En general, si llamamos $C^{n+1}$ al conjunto de combinaciones válidas obtenidas
a partir de $C^n$ y aplicamos el razonamiento anterior podemos acotar $C^{\ast}$
por:

\[ 4 \cdot C_6^2 \cdot 4 \cdot C_5^2 \cdot 4 \cdot C_4^2 \cdot 4 \cdot C_3^2 \cdot 4 \cdot C_2^2= 2764000\]

De esta forma, podemos definir $T:\mathbb{N} \to \mathbb{N}$:
\[T_n(i)=n^i\prod_{k=0}^{i}C_{i-k}^{2}\]

donde $T_n(|C|)$ devuelve el tamaño de los generables a partir de $C$ con $n$
operaciones. La división no siempre es válida, por lo que tenemos que:

\[T_3(|C|) < C^{\ast} < T_4(|C|)\]
