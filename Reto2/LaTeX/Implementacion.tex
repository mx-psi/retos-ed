\section{Implementación}

En la carpeta \texttt{Implementaciones} pueden encontrarse implementaciones
en C++ de los tres algoritmos y una modificación del primero que obtiene las
combinaciones mágicas. Incluimos un \texttt{Makefile} para facilitar
su compilación.
Para compilar solo el algoritmo \texttt{n} basta con usar la orden
\texttt{make Algoritmo[n].out} sin corchetes.\\

Para la ejecución de cada algoritmo existen dos opciones:

\begin{itemize}
  \item Pasar \textbf{7} argumentos indicando los números iniciales y la
  solución (en este orden).
  \item No pasar ninguno (en cuyo caso se generará un ejemplo aleatorio).
\end{itemize}

Los programas muestran las operaciones hasta alcanzar la mejor aproximación
(salvo el segundo, que si no encuentra el resultado exacto solo mostrará
el resultado más cercano sin desplegar las operaciones que llevaron al
mismo) y el tiempo invertido en hallar la solución. \\

El programa para obtener las combinaciones mágicas, que se compilará con
el nombre \texttt{Algoritmo1magic.out}, las devuelve encerradas en llaves
y separadas por saltos de línea en salida estándar (puede redirigirse la
salida a archivo usando \texttt{Algoritmo1magic.out\ >\ nombrearchivo})
y adjunta el tiempo de ejecución.
